\documentclass{nsf_proposal}
%%%%%%%%%%%%%%%%%%%%%%%%%%%%%%%%%%%%%%%%%%%%%%%%%%%%%%%%%%%%%%%%%%%%%%%%
% LaTeX template to generate an NSF proposal
%
% First version by: Stefan Llewellyn Smith, Sarah Gille, others.
%
% Additions by: Ronni Grapenthin, New Mexico Tech.
%
% This version maintained/modified by: Jeremy A. Gibbs, Univ. of Utah
%
% This template is free source code. It comes without any warranty, to
% the extent permitted by applicable law. You can redistribute it and/or
% modify it under the terms of the Do What The Fuck You Want To Public
% License, Version 2, as published by Sam Hocevar. See
% http://www.wtfpl.net for more details.
%%%%%%%%%%%%%%%%%%%%%%%%%%%%%%%%%%%%%%%%%%%%%%%%%%%%%%%%%%%%%%%%%%%%%%%%
\usepackage[T1]{fontenc}
\usepackage[latin1]{inputenc}
\usepackage[amssymb]{SIunits}
\usepackage{latexsym}
\usepackage{multirow}
\usepackage{wrapfig}
\usepackage{amsmath, amsthm, amssymb}
\usepackage{amsfonts}
\usepackage[format=plain,indention=0cm, font=small, labelfont=bf]{caption}
\usepackage{longtable}
\usepackage{fancyhdr}
\usepackage[pdftex]{graphicx}
\usepackage[pdftex,
	    colorlinks,
	    pdfstartview=FitH,
	    linkcolor=black,
	    citecolor=black,
	    urlcolor=black,
	    filecolor=black
	    ]{hyperref}
\usepackage{lscape}
\usepackage{floatrow}
\usepackage{enumerate}
\usepackage{enumitem}
\usepackage{tabularx}
\usepackage{ragged2e}
\usepackage[table]{xcolor}
\usepackage{comment}
\usepackage{ifthen}

\newcommand{\final}{0}
\newcommand{\todo}[1]{\footnotesize\color{red}[TODO\@: #1]}
\newcommand{\jdo}[1]{{\footnotesize\color{blue}[John: #1]}}
\ifthenelse{\equal{\final}{1}}{\renewcommand{\todo}[1]{}\renewcommand{\jdo}[1]{}

\newcolumntype{Y}{ >{\RaggedRight\arraybackslash}X}
\newcommand\T{\rule{0pt}{2.6ex}}
\newcommand\B{\rule[-1.2ex]{0pt}{0pt}}

\renewcommand{\refname}{References Cited}

\newcommand{\degrees}{$\!\!$\char23$\!$}

\def\rrr#1\\{\par
\medskip\hbox{\vbox{\parindent=2em\hsize=6.12in
\hangindent=4em\hangafter=1#1}}}

\renewcommand\baselinestretch{1}
\pagestyle{empty}

%\setlength{\parindent}{0pt}
\setlength{\parskip}{1ex}

%!TEX root =  proposal.tex

\usepackage[moderate,bibnotes,charwidths,mathdisplays,indent]{savetrees}

%\usepackage[textsize=tiny]{todonotes} % package for cool todonotes in the margin
%\usepackage[normalem]{ulem} % package that allows text to be striked out
\usepackage{titlesec}

\titlespacing*{\section}
{0pt}{2.5ex plus 1ex minus .2ex}{1.3ex plus .2ex}
\titlespacing*{\subsection}
{0pt}{2.5ex plus 1ex minus .2ex}{1.3ex plus .2ex}


\usepackage{amsmath}
\usepackage{amsthm}
\usepackage{amstext}
\usepackage{amsfonts}
\usepackage{mathrsfs}
\usepackage{xcolor}
\usepackage{tikz}
\usepackage{cleveref}
\usepackage{subcaption}
\usepackage{cite}
\usepackage{pgfplots}
\pgfplotsset{compat=1.3}
\usepackage{url}
\urlstyle{tt}
\usepackage{xspace}
%\usepackage{algorithm}
%\usepackage[noend]{algpseudocode}
%\usepackage{siunitx}
\usepackage{booktabs}
\usepackage{multirow}
\usepackage{comment}
\usepackage[title]{appendix}
\usetikzlibrary{external}
%\usepackage{enumitem}
\usepackage{balance}
\usepackage{pgfgantt}

\iffalse
\usepackage{xr}
% In your preamble

\makeatletter
\newcommand*{\addFileDependency}[1]{% argument=file name and extension
  \typeout{(#1)}
  \@addtofilelist{#1}
  \IfFileExists{#1}{}{\typeout{No file #1.}}
}
\makeatother

\newcommand*{\myexternaldocument}[1]{%
    \externaldocument{#1}%
    \addFileDependency{#1.tex}%
    \addFileDependency{#1.aux}%
}
\myexternaldocument{supplementary}
\fi



\usepackage{graphicx}

\usepackage{svg}

%\usepackage{amssymb}

%!TEX root =  proposal.tex


%% MAB: for proposing text
\definecolor{coolblack}{rgb}{0.0, 0.18, 0.39}
\definecolor{darkcerulean}{rgb}{0.03, 0.27, 0.49}
\newcommand{\propose}[1]{{\color{darkcerulean}#1}}

% MAB: this is the macro we can use to enumerate research problems
\newtheorem{eval}{Evalulation}
\newtheorem{prob}[eval]{Research Problem}
\newtheorem{rproblem}[eval]{Research Problem}
\newtheorem{proofofconcept}[eval]{Proof of Concept}

\newcommand{\evalref}[1]{Evaluation~\ref{#1}}
\newcommand{\evalreftwo}[2]{Evaluations \ref{#1} and~\ref{#2}}
\newcommand{\evalrefthree}[3]{Evaluations \ref{#1}, \ref{#2},             and~\ref{#3}}
\newcommand{\evalreffour}[4]{Evaluations \ref{#1}, \ref{#2}, \ref{#3},                   and~\ref{#4}}


\newcommand{\errbits}{\log (1/\epsilon)}
\newcommand{\opt}{n\errbits}
\renewcommand{\epsilon}{\varepsilon}

% \newcommand{\defn}[1]{\textit{#1}}
\newcommand{\defn}[1]{{\textit{\textbf{\boldmath #1}}}\xspace}
\newtheorem{theorem}{Theorem}
\newtheorem{lemma}[theorem]{Lemma}
\newtheorem{corollary}[theorem]{Corollary}
\newtheorem{claim}[theorem]{Claim}
\newtheorem{proposition}[theorem]{Proposition}
\newtheorem{assumption}[theorem]{Assumption}
\newtheorem{definition}[theorem]{Definition}
\newtheorem{invariant}[theorem]{Invariant}
\newtheorem{observation}[theorem]{Observation}

\ifdefined\textln\relax\else\newcommand{\textln}[1]{#1}\fi

%\makeatletter
%\algrenewcommand\ALG@beginalgorithmic{\footnotesize}
%\makeatother


\definecolor{magenta4}{rgb}{0.5625,0,0.5625}
\definecolor{green4}{rgb}{0,0.5625,0}
\definecolor{orange4}{rgb}{0.98,0.31,0.09}
\definecolor{powderblue}{rgb}{0.69,0.88,0.9}

\iftrue
\newcommand{\mab}[1]{{\scriptsize \textcolor{blue}{Michael: {#1}}}}
\newcommand{\mfc}[1]{{\scriptsize \textcolor{orange4}{Martin: {#1}}}}
\newcommand{\john}[1]{{\scriptsize \textcolor{pink}{John: {#1}}}}
\newcommand{\prashant}[1]{{\scriptsize \textcolor{cyan}{Prashant: {#1}}}}
\fi


% see http://latexcolor.com to define more colors

\newcommand{\fixme}[1]{\todo[size=\tiny]{#1}}

\newcommand{\TODO}{{\color{red}\bf TODO}}


\newcommand{\qf}{quotient filter\xspace}
\newcommand{\cqf}{counting quotient filter\xspace}
\newcommand{\bloom}{Bloom filter\xspace}
\newcommand{\btree}{B-tree\xspace}
\newcommand{\btrees}{B-trees\xspace}
\newcommand{\bplustree}{B$^+$-tree\xspace}
\newcommand{\bplustrees}{B$^+$-trees\xspace}
\newcommand{\proc}{\texttt}


\newcommand{\kmer}{$k$-mer\xspace}
\newcommand{\Kmer}{$K$-mer\xspace}
\newcommand{\kmers}{$k$-mers\xspace}
\newcommand{\Kmers}{$K$-mers\xspace}


\newcommand{\poly}[1]{\textrm{poly}{(#1)}}
\newcommand{\polylog}[1]{\textrm{polylog}{(#1)}}

\newcommand{\para}[1]{\smallskip\noindent\textbf{#1.}}

\newcommand{\punt}[1]{}
