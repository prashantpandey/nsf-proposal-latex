%!TEX root =  proposal.tex

\section*{Post-Doctoral Researcher Mentoring Plan}
\label{postdoc}



PI Bender's work on this five-year project will begin by focusing on primarily on algorithms and
transition to systems building, so Bender is requesting a postdoc in years four and five. The post-doc will be mentored by Bender (Stony Brook),
with substantial interaction with Farach-Colton (Rutgers).  Farach-Colton has
mentored post-docs who are now faculty members at Princeton, Carnegie
Mellon, UC San Diego, and Harvey Mudd College, as well as one who is a
Senior Research Scientist at the National Institutes of Health.

Bender hopes to hire the post-doc from among the PhD students that PIs Owens or Pandey advise.  The idea will be to transition students who have focused primarily on systems into being true systems/theory hybrids.  Both Bender and Farach-Colton have extensive experience mentoring systems/theory hybrids, including PI Pandey, who was an advisee of Bender's.  In addition, the systems experience of the post-doc will facilitate Bender's transition in this project from algorithms to systems.  Bender and Farach-Colton have experience mentoring post-docs who build systems.  Specifically, David Tench was a post-doc of Bender's and is now a CRA post-doc with Farach-Colton.  He has lead several implementation efforts in using sketches for distributed graph processing. 

Post-docs will interact with mentors in weekly project meetings
and, additionally,  whenever  necessary, e.g., to discuss professional development or course-preparation topics.

Post-docs will in turn mentor graduate and undergraduate students
working on this project.  Post-docs will participate in weekly project
meetings and will meet with the students on the project independently,
as needed.  We envision the post-docs roles as similar to a senior PhD
student: they will be expected to work with the PIs to develop a
vision for the overall project and to help the students stay on track
with that vision. 

These post-docs will gain experience in the grant preparation and
management process.  They will work closely with PIs to manage all
aspects of proposals and grants, including budgeting, reporting, and
regulatory requirements. They will be able to serve as co-PIs and
gain actual experience in grantsmanship.

The post-docs will have the opportunity to create their own
specialized courses on GPU data structures and related
topics.  Stony Brook offers special topics classes at both the
graduate and undergraduate level.  Furthermore, the Stony Brook
undergraduate 
honors program, which PI Bender and co-PI Johnson co-direct, provides
a bright and eager pool of undergraduate students interested in
advanced topics.

We have budgeted travel funds for the post-docs on this grant.  The
post-doc will have access to funds for at least one trip per year,
although they may have extra funds for travel if the PIs or graduate
students choose to travel less than budgeted.  We will encourage the
post-doc to travel to Rutgers, among other locations, to
collaborate with researchers on the problems addressed by this
project.  PI Farach-Colton lives half-way between the two
universities, further facilitating meetings.

The post-doc funded by this project will work with their mentors to create
statements of Job Responsibilities and Goals, that outline what
specific work they are expected to undertake for the project, identify
specific career-development objectives, and establish additional
activities that support those objectives. Initial versions of these
documents will be reviewed by all project PIs and will be used to
assess mentoring progress in quarterly reviews.
