%!TEX root =  proposal.tex

\section*{Facilities ---  UMD}

Dr. Patro is a member of the Center for Bioinformatics and Computational Biology, which provides all associated labs with access to a shared compute cluster with the following specifications:

\begin{itemize}
\item CBCB partition in Nexus which includes 704 Zen3 cores, 44TB of main memory and 44TB of local storage distributed over 22 servers.
\item Pair of 64 port 100Gb Mellanox SN4600 leafs participating in our 100Gb Nexus research data network. 
\item Two 48 port 10Gb downstream/40Gb upstream leaf nodes in the Big Switch research data network.
\item DellEMC Isilon H400 NFS and CIFS based storage cluster with 320TB of space across 6 nodes.
\end{itemize}

Additionally, as a member of UMIACS and as faculty in the Computer Science Department (CSD), Dr. Patro’s lab has access to a broader variety of hardware available via the Tron partition of the larger Nexus cluster. The Nexus cluster is a next generation centralized scheduler built on SLURM. The resources in the Tron partition are available to all UMIACS and CSD faculty and their students along with allocations for CSD academic teaching support.

\begin{itemize}
\item Tron partition of 62 nodes with 1,344 CPUS, 10.7 TB of memory and 50/100Gb Ethernet connectivity.  In these 62 nodes there are 48 Nvidia A6000, 128 Nvidia A5000, and 160 Nvidia A4000 GPUs for a total of 336 GPUs with more than 70,000 TensorFLOPs of aggregate TF32 performance.  
\item Mellanox SN4600 100Gb Ethernet spine/leaf network with over 512 100Gb Ethernet leaf ports.  This network allows our other centers and labs to connect resources to the Nexus and take advantage of this centralized computing core of UMIACS.
\item VAST Data system with 500TB of NVMe storage available via multiple 100Gb Ethernet links over NFS.  The system also supports GPUDirect Storage for Nvidia GPUs to allow direct GPU access to bypass system memory and CPU and directly access storage resources across the network using RDMA over Converged Ethernet.
\end{itemize}

Likewise, as a member of UMIACS, Dr. Patro and his students have access to the OpenLAB infrastructure, which consists primarily of a number of services and resources that are commonly available to all members of UMIACS, including:

\begin{itemize}
    \item GitLab source code repository with issue tracking, code review and continuous integration options.

    \item Obj is a Ceph based S3 compatible storage system with 1400 cores, 4TB memory and  6 PB of total capacity.  This provides support for users to hold and distribute large scale datasets including serving buckets in Lab Groups or publicly.  Faculty purchase nodes within the platform for longer term durable data storage.

    \item An enterprise virtualization platform based on VMware, 50TB DellEMC ME5024 storage area network, and 75TB DellEMC Isilon H400 network attached file systems.

    \item A partition in the Nexus cluster equipped with 736 cores.  The cluster is regularly updated with a sampling of the latest processor and GPU technologies and retains multiple generations of processor technologies to facilitate the development of new high performance computing algorithms.
\end{itemize}

Finally, Dr. Patro’s lab has access to a dedicated server running Ubuntu linux (20.04) with 44 Intel(R) Xeon(R) CPU E5-2699 cores, 512GB of RAM, and 36TB of storage.

\subsubsection*{Other facilities \& resources at UMD}

Dr. Patro is a faculty member in the Computer Science Department (CSD) at the University of Maryland, College Park (UMD), and is also a member of the University of Maryland Institute for Advanced Computer Studies (UMIACS) and the Center for Bioinformatics and Computational Biology (CBCB).  As a member lab of the CBCB, Dr. Patro’s graduate students all have office space in the Brendan Iribe Center for Computer Science and Engineering, and the lab space resides just down the hall from Dr. Patro’s office. Likewise, Dr. Patro has his own office, with a desktop computer, telephone, and internet connectivity. Dr. Patro and his students all have access to the UMD library system, which provides access to articles from a variety of academic publishers. Likewise, they have access to the extensive collections available via the Hornbake, McKeldin and STEM Libraries, the last of which focuses on highly technical content. These facilities can be used to support various objectives of the proposed study. As members of the University community, Dr. Patro and his students have access to the flagship Deepthought2 compute cluster, which is a university-wide resource maintained by the Division of Information Technology. 