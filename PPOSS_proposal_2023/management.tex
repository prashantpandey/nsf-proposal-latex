%!TEX root =  proposal.tex

\section*{Management and Coordination}


The PIs on this project have a long history of collaboration. It is because of our many years of collaboration— across many areas of algorithms, system building, computational biology applications, and technology transfer—that we were able to be so productive even during the covid years.
%
The research group will have standing meetings on a weekly basis.  In addition, we will plan retreats, either just before or after a conference we are all attending, or virtually.  We already have experience organizing virtual retreats in our other multi-institutional research groups, and they have gone well. {\bf Each budget has money budgets for travel to retreats.}

In the past, the PIs have decided on research direction and priority by consensus. We are a small enough group and in sufficiently frequent contact that we expect this project to advance similarly. The management of this proposal will follow the successful formula from the past.

\noindent
{\bf Team.}
This work requires a team with cross-cutting expertise, including experience with cross-layer system prototyping and theory. Our team of Michael Bender (SBU, Theory and HPC), Martin Farach-Colton (NYU, Theory), John Owens (UC Davis, Systems and HPC), Rob Patro (UMD, Theory and Computational Biology) and Pandey (Utah, Systems, HPC, and Computational Biology) has precisely these rare qualifications, and a track-record of successful collaboration.
%
% Other senior personnel, who will collaborate on work during the next phase of research include Kathy Yelick (Berkeley, HPC, Systems, Applications) and Leonid Oliker (Berkeley, HPC, Systems, Applications).

\noindent
{\bf A record of collaboration.}
The PIs have an extensive track record of working together. PIs Bender and Pandey have written over 20 papers together, many of them on data structures for computational biology.  PI Farach-Colton has also written many papers on computational biology.  PIs Farach-Colton and Owens have written papers dynamic GPU data structures.  PIs Bender and Farach-Colton have written almost 80 papers together and have also started a database performance company, Tokutek, to commercialize their research.  PIs Pandey and Patro have written more than 12 papers on computational biology applications. In addition, we have numerous grants together, we've taught tutorials all over the world together, and have co-advised each other's students. 

\noindent
{\bf Software development and deployment.} The software will be developed iteratively; the outcome of each iteration will be a tested, ensuring executability with partial but stable functionality aided by runtime verification methods. All software and associated materials will be released under an appropriate open-source license (previous licenses used by the PIs include LGPL, BSD, and Apache). Regression testing will be performed weekly. A defect tracking system will be employed. The repository will contain all code, regression tests, design documents, API specifications, manuals, and tutorials.
