\section{Broader Impacts}
\label{broadimpacts}

The primary broader impact of this work is to enhance the capability of bioinformatics applications to perform computations at the largest scales. Both accelerated computation (allowing quicker feedback and more experiments) and larger computation potentially accelerate the process of scientific discovery. As MIT professor Bonnie Berger notes (with respect to computing drug-target interactions, but more generally as well)~\cite{Krasilnikova:2022:DDD}, ``the data's growing enormously. It's been very difficult to handle, and it will keep growing, but it has huge opportunities.'' \john{Prashant: if there's a good societal benefit here, say, something about disease, could you add?}

Secondly, we expect that a high-quality library of distributed, scalable GPU data structures will find widespread utility in other application domains. Although applying our library to such domains is outside of the scope of this project, we see numerous application domains where this work can have a direct impact. For instance, PI Owens found that multi-GPU graph computation on GPUs without scalable data structures is largely limited by GPU-to-GPU bandwidth~\cite{Pan:2017:MGA} and has proposed complex mechanisms like asynchronous execution and on-GPU communication management~\cite{Chen:2022:SIP} to alleviate this bottleneck. The distributed hash tables in Section~\ref{sec:hash-tables} can directly address this bottleneck.

Beyond technical broader impacts, we propose two outreach impacts that we hope will bridge the gaps between computational biologists, CS theorists, and CS systems researchers:

\begin{enumerate}[noitemsep]
  \item We will develop a tutorial on our software and techniques that is suitable for presentation at computational biology conferences (e.g., ACM RECOMB, ISMB, or PSB) and database conferences. Our primary goal here is enabling attendees to use both our computational-biology applications and the underlying data structures library, and secondarily to understand the design decisions and research advances that underpin them.
  \item We will propose, and lead, a week-long Dagstuhl seminar to bring together researchers from these three communities, with the same specific focus as this proposal: large-scale applications in computational biology.
\end{enumerate}

Finally, the PIs commit to furthering their strong records of working with students that are traditionally underrepresented in computing fields. Owens works with two UC programs, UC LEADS~\cite{UCLeads:2003:WEB} and the UC Davis McNair Scholars Program~\cite{McNair:2003:WEB}.  The PIs currently advise six female Ph.D.\ and three female M.S.\ students and have a long history of supervising students and postdocs from underrepresented group.  
PI Farach-Colton is Hispanic and has recruited many Hispanic students into PhD programs.


