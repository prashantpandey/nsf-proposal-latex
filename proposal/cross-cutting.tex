\section{Cross-cutting Concerns}

% The objective of the Principles and Practice of Scalable Systems (PPoSS) program, therefore, is to build a community of researchers who will work symbiotically to perform basic research on scalability, performance, and correctness and accuracy of modern applications, systems, and toolchains built on heterogeneous architectures. PPoSS expects coordinated progress at the intersection of multiple disciplines including, but not limited to, computer architecture, HPC, programming languages and compilers, machine programming, security and privacy, systems, and theory and algorithms. All the cross-cutting concerns listed below must be addressed from the outset in all aspects of systems design and implementation, and must be tackled with respect to the full hardware and software stack:

% Scalability and performance (including, but not limited to, portability, resource usage, and energy efficiency);
% Correctness and accuracy (including, but not limited to, emerging techniques for programming, analysis, testing, debugging, probabilistic reasoning and inference, verification); and
% Heterogeneous architectures (including, but not limited to, systems with many cores and heterogeneous computing components such as Graphics Processing Units (GPUs), Tensor Processing Units, Field-Programmable Gate Arrays (FPGAs), continually changing memory systems (e.g., Non-Volatile Random Access Memories).
% The intent is that these efforts will foster the development of principles that lead to rigorous and reproducible methodologies for the design and implementation of large-scale systems while simultaneously incorporating the cross-cutting concerns listed above.
