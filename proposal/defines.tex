%!TEX root =  proposal.tex

\makeatletter
\def\thm@space@setup{\thm@preskip=1pt
\thm@postskip=1pt}
\makeatother

%% MAB: for proposing text
\definecolor{coolblack}{rgb}{0.0, 0.18, 0.39}
\definecolor{darkcerulean}{rgb}{0.03, 0.27, 0.49}
\newcommand{\propose}[1]{{\color{darkcerulean}#1}}

% MAB: this is the macro we can use to enumerate research problems
\newtheorem{eval}{Evalulation}
\newtheorem{prob}[eval]{Research Problem}
\newtheorem{rproblem}[eval]{Research Problem}
\newtheorem{proofofconcept}[eval]{Proof of Concept}

\newcommand{\evalref}[1]{Evaluation~\ref{#1}}
\newcommand{\evalreftwo}[2]{Evaluations \ref{#1} and~\ref{#2}}
\newcommand{\evalrefthree}[3]{Evaluations \ref{#1}, \ref{#2},             and~\ref{#3}}
\newcommand{\evalreffour}[4]{Evaluations \ref{#1}, \ref{#2}, \ref{#3},                   and~\ref{#4}}


\newcommand{\errbits}{\log (1/\epsilon)}
\newcommand{\opt}{n\errbits}
\renewcommand{\epsilon}{\varepsilon}

% \newcommand{\defn}[1]{\textit{#1}}
\newcommand{\defn}[1]{{\textit{\textbf{\boldmath #1}}}\xspace}
\newtheorem{theorem}{Theorem}
\newtheorem{lemma}[theorem]{Lemma}
\newtheorem{corollary}[theorem]{Corollary}
\newtheorem{claim}[theorem]{Claim}
\newtheorem{proposition}[theorem]{Proposition}
\newtheorem{assumption}[theorem]{Assumption}
\newtheorem{definition}[theorem]{Definition}
\newtheorem{invariant}[theorem]{Invariant}
\newtheorem{observation}[theorem]{Observation}

\ifdefined\textln\relax\else\newcommand{\textln}[1]{#1}\fi

%\makeatletter
%\algrenewcommand\ALG@beginalgorithmic{\footnotesize}
%\makeatother


\definecolor{magenta4}{rgb}{0.5625,0,0.5625}
\definecolor{green4}{rgb}{0,0.5625,0}
\definecolor{orange4}{rgb}{0.98,0.31,0.09}
\definecolor{powderblue}{rgb}{0.69,0.88,0.9}

\iftrue
\newcommand{\mab}[1]{{\scriptsize \textcolor{blue}{Michael: {#1}}}}
\newcommand{\mfc}[1]{{\scriptsize \textcolor{orange4}{Martin: {#1}}}}
\newcommand{\john}[1]{{\scriptsize \textcolor{pink}{John: {#1}}}}
\newcommand{\prashant}[1]{{\scriptsize \textcolor{cyan}{Prashant: {#1}}}}
\newcommand{\rob}[1]{{\scriptsize \textcolor{cyan}{Rob: {#1}}}}
\fi
\iffalse
\newcommand{\mab}[1]{}
\newcommand{\mfc}[1]{}
\newcommand{\john}[1]{}
\newcommand{\prashant}[1]{}
\newcommand{\rob}[1]{}
\fi

% see http://latexcolor.com to define more colors

\newcommand{\fixme}[1]{\todo[size=\tiny]{#1}}
\newcommand{\draft}[1]{{\textcolor{red}{#1}}}

\newcommand{\TODO}{{\color{red}\bf TODO}}


\newcommand{\qf}{quotient filter\xspace}
\newcommand{\cqf}{counting quotient filter\xspace}
\newcommand{\bloom}{Bloom filter\xspace}
\newcommand{\btree}{B-tree\xspace}
\newcommand{\btrees}{B-trees\xspace}
\newcommand{\bplustree}{B$^+$-tree\xspace}
\newcommand{\bplustrees}{B$^+$-trees\xspace}
\newcommand{\proc}{\texttt}

\newcommand{\acro}[1]{{#1}}

\newcommand{\kmer}{$k$-mer\xspace}
\newcommand{\boldkmers}{{\boldmath $k$-mers}\xspace}

\newcommand{\Kmer}{$K$-mer\xspace}
\newcommand{\kmers}{$k$-mers\xspace}
\newcommand{\Kmers}{$K$-mers\xspace}


\newcommand{\poly}[1]{\textrm{poly}{(#1)}}
\newcommand{\polylog}[1]{\textrm{polylog}{(#1)}}

\newcommand{\para}[1]{\smallskip\noindent\textbf{#1.}}
\setlength{\parskip}{3pt}


\newcommand{\punt}[1]{}