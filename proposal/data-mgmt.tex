
%!TEX root =  proposal.tex

%\section*{Special Information and Supplementary Documentation}

\section*{Data Management Plan}

\label{sec:datamgmt}

\begin{enumerate}
\item [1] \textbf{Expected Data}

The primary type of data to be produced by this project will be
software artifacts and measurements of these software artifacts.
The project will also produce papers, presentations, lecture notes,
and other artifacts.
\item [2] \textbf{Data Format}

During the project, software source code will be stored using git, an industry-standard
version control system, hosted on \url{github.com} and
backed up on the cloud and
on servers across partner institutions. While Github is commonly used and in our experience is highly reliable, it is commercial and has no guarantees of long-term archiving. Thus we have identified Zenodo as a backup repository for code, data, and documentation. Among other desirable features, Zenodo can directly import from Github and offers DOIs for citation purposes.
%which is backed up on multiple lab servers in different wings
%of the Stony Brook computer science department building, as well as on
%
Data measurements
will also be stored electronically in git,
in plain text or typeset using \LaTeX, or in Google
Drive, as a spreadsheet.

We do not expect interested parties to need this data until after publication.

\item [3] \textbf{Access to Data}

We will create and maintain a project web page. % at \url{betrfs.org}
All publicly releasable data sets,
papers, presentations, software releases, and other results of this
project will be available on the website.

Collected data will also be available on the project website as published
papers and supplements, as appropriate, and after peer-reviewed
publication.
After publication, software produced for this project will be
made available on Github.
%at \url{github.com/oscarlab/betrfs}.


\item [4] \textbf{Data Sharing Practices and Policies}

Our policy is to share data with an interested party as long as it does not
harm potential stakeholders.  Generally, this is (1) after intellectual property
disclosures have been filed with the university and there are no pending patent concerns,
(2) when it is unlikely the students' ability to publish the work would be harmed (scooped) by the data release
and (3) when the release would not violate any software licenses or other laws.

\item [5] \textbf{Policies for Re-Use, Re-Distribution}

The use of posted system measurements will be unencumbered.
Modified software will be subject to the original copyright and license terms.

Original software will be copyrighted by the authors and released
using the MIT license or another open-source license such as the
GPL\@.  The MIT license allows other researchers, students, open
source projects, and commercial entities to incorporate the software
into their projects.  The GPL is more restrictive, since it requires
that derived works publish their source code.  The choice of license
will depend partly on where the code is used.  For example, Linux
kernel contributions may require GPLv2. Source code documentation will
be collocated and released with software, governed by an open source
license, and hosted on public repositories. This is the current
practice of the collaborators.

Because we will be modifying real-world software as a research prototype,
redistribution of software may be subject to legal restrictions
of the original software licenses; we will respect these restrictions and
use the least restricted software when possible.


\item [6] \textbf{Archiving of Data}

Electronic data will be archived internally on multiple computers
in different wings of the computer science department,
minimizing the risk of fire, water, or electrical damage.
We will take incremental snapshots and perform routine integrity checks,
maximizing the ability to recover corrupted data.

All data will be kept for at least 5 years after publication; likely longer,
depending on its size and outside interest.

All data published to the internet will be indexed by search engines, such as Google,
backed up internally (as described above),
and externally archived (e.g., \url{archive.org}).

\end{enumerate}

\paragraph*{Roles and responsibilities.}
Each PI %(Farach-Colton at Rutgers, Jannen at Williams, Johnson at SBU,
%Porter at UNC, and Yuan at Farmingdale)
will serve as Data Management
Officer responsible for implementing the data management plan at their
site. Data management activities will be reviewed in annual project
meetings and accomplishments and/or plan modifications will be
reported annually to the NSF\@.
