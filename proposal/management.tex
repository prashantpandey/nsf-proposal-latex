\section*{Management and Coordination}

% Every LARGE proposal must contain a clearly labeled "Management and Coordination Plan", which includes: 1) the specific roles of the PI, co-PIs, other senior personnel, and paid consultants at all organizations involved to demonstrate that the project personnel have distinct but complementary expertise at different levels of the hardware/software stack, and in at least four of the program's research areas; 2) how the project will be managed across organizations and expertise; 3) identification of the specific coordination mechanisms that will enable cross-organization and/or cross-expertise scientific integration and achieve synergy within the team; and 4) pointers to the budget line items that support these management and coordination mechanisms.



The PIs have an extensive track record of working together.  PIs Bender and Pandey have written over 20 papers together, many of them on data structures for computational biology.  PI Farach-Colton has also written many papers on computational biology.  PIs Farach-Colton and Owens have written papers dynamic GPU data structures.  PIs Bender and Farach-Colton have written almost 80 papers together and have also started a database performance company, Tokutek, to commercialize their research.  In addition, we have numerous grants together, we've taught tutorials all over the world together, and have co-advised each other's students. 

The research group will have standing meetings on a weekly basis.  In addition, we will plan retreats, either just before or after a conference we are all attending, or virtually.  We already have experience organizing virtual retreats in our other multi-institutional research groups, and they have gone well.

Because of our success in the past and our extensive collaboration, decision making will be by consensus.  