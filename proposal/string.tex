\section{String data structures}

%
String pattern matching is one of the most common and most computationally intensive problem in computational biology.
%
String pattern matching is the problem of counting or locating occurrences of a query text pattern $P$ within a large database text $T$. While not limited to the analysis of biological sequences, string pattern matching is integral to many tasks in computational biology, including mapping sequence reads to a reference genome~\cite{li2009fast,langmead2009ultrafast}, sequence to graph alignment~\cite{Jain2020}, taxonomic classification~\cite{menzel2016fast,kim2016centrifuge}, sequencing error correction~\cite{huang2017efficient}, and seeding for sequence alignments~\cite{buchfink2015fast,altschul1990basic,steinegger2017mmseqs2}.

The need for high-throughput pattern matching in computational biology has motivated numerous approaches including hashing, lookup tables, suffix arrays~\cite{manber1993suffix}, and compressed suffix array data structures such as the FM-index~\cite{ferragina2000opportunistic}. 
FM-index is based on Burrows-Wheeler transform~\cite{burrows1994block}. The Burrows-Wheeler transform (BWT) is a reversible text transform that was originally proposed for lossless data compression.

% Strings and subsequences appear in a plethora of computational biology problems because the basic types of DNA, RNA, and protein molecules can be represented as strings—pieces over a given alphabet of DNA as strings over the alphabet {A,C, G, T } (representing the four bases adenine, cytosine, guanine, and thymine, respectively), pieces of RNA as strings over the alphabet {A,C, G, U} (with uracil replacing thymine) and proteins as strings over an alphabet of 20, corresponding to the 20 amino acid residues.

% Let $A$ be a finite alphabet consisting of a set of characters (or symbols).  The cardinality of the alphabet denoted by $|A|$ expresses the number of distinct characters in the alphabet.  A string or word is an ordered list of zero or more characters drawn from the alphabet. A word or string $w$ of length $n$ is represented by $w[1 \ldots n] = w[1],w[2], \ldots ,w[n]$, where $w[i] \in A \; for \; 1 \leq i \leq n$ and $|w|$ denotes the length of $w$. The empty word is the empty sequence (of zero length) and is denoted by $\phi$.  A list of characters of $w$, appearing in consecutive positions, is called a \emph{substring} of $w$, denoted by $w[i \ldots j]$, where $i$ and $j$ are the starting and ending positions, respectively. If the substring starts at position 1, then it is called a \emph{prefix}, whereas if it ends at position $n$, then it is called a \emph{suffix} of $w$. However, an ordered list of characters of $w$ that are not necessarily consecutive is called a \emph{subsequence} of $w$.

Use of the FM-index across applications is due to its fast performance and low memory footprint. Unfortunately, its adoption is likely limited by the lack of an optimized and lightweight FM-index library; the only robust, currently maintained FM-index implementations are SeqAn3 library~\cite{Reinert2017} and AWFM-index~\cite{Anderson2021}.

Furthermore, FM-index is a static data structure. It saves space at the cost of 

\begin{rproblem}
Prove a lower-bound for space and run time to support updates in FM-index.
\end{rproblem}


\begin{rproblem}
Prove lower bounds for space and run time for sequence to graph alignment.
\end{rproblem}

\begin{rproblem}
Design and build a sequence to graph alignment library for GPUs.
\end{rproblem}


\begin{rproblem}
Design and build a sequence to graph alignment library for GPUs in distributed-memory when the sequence graph is stored across multiple GPUs.
\end{rproblem}





