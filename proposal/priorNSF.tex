\section{Results from Prior NSF Support}

\noindent \emph{\underline{Person One}}: No NSF support in the past five years \newline

\noindent The most relevant prior NSF award to the proposed project for
\underline{Person Two} (Co-PI) is: (a) NSF PDM \#\#\#\#\#\#\#, \$000,000,
MM/DD/YY to MM/DD/YY; (b) Title: Super Cool Project That Got Funded; (c)
Accomplishments related to the \textbf{intellectual merit} of this research project
include something something. The \textbf{broader impacts} include outreach at many
levels. Something Something. To date, the grant has funded one post-doc and 1000
graduate students. The project has also involved 500 undergraduate students. (d)
To date this project has resulted in 100 conference presentations, one million
journal publications (cite them) with one under review (cite it) and two in
preparation with well-developed drafts.

For PI Owens, the NSF award whose outcomes most influenced this grant was the recently completed NSF Algorithms in the Field award CCF-1637442, ``AitF: Collaborative Research: Theory and Implementation of Dynamic Data Structures for the GPU'', with Mart\'{i}n Farach-Colton (Rutgers). This work focused on the design and implementation of GPU-friendly, dynamic, general-purpose data structures.

\paragraph{Intellectual Merit.} In this AitF award, the PIs designed and built numerous dynamic data structures that marked their first implementation on GPUs: linked lists and hash tables~\cite{Ashkiani:2018:ADH}, log-structured merge trees~\cite{Ashkiani:2018:GLA}, B-trees~\cite{Awad:2019:EAH}, quotient filters~\cite{Geil:2018:QFA}, and dynamic graphs~\cite{Awad:2020:DGO}. RXMesh's mesh data structure~\cite{Mahmoud:2021:RAG} was a result of this award. As well, the PIs recently designed and implemented the fastest GPU static hash tables~\cite{Awad:2023:AAI}.

\paragraph{Broader Impacts.} Each of the abovementioned projects is available as open-source software, a significant reason why this work is highly cited. The dynamic graphs were integrated into Gunrock~\cite{Awad:2020:DGO,Wang:2017:GGG}, which subsequently was selected as the performance reference for the DARPA HIVE program and in turn integrated into NVIDIA's open-source RAPIDS framework for data science. Finally, NVIDIA's cuCollections library currently only supports static and dynamic maps but these were both influenced by the PIs' work and we expect as this library grows, it will incorporate the abovementioned work either directly or as the basis for NVIDIA's implementation and improvements.
