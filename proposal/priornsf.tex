\section{Results from Prior NSF Support}

PI Pandey has no NSF awards yet.

For PI Owens, the NSF award whose outcomes most influenced this grant was the recently completed NSF Algorithms in the Field award CCF-1637442, ``AitF\@: Collaborative Research: Theory and Implementation of Dynamic Data Structures for the GPU'', with PI Farach-Colton. This work focused on the design and implementation of GPU-friendly, dynamic, general-purpose data structures.
\textbf{Intellectual Merit.} In this AitF award, the PIs designed and built numerous dynamic data structures that marked their first implementation on GPUs: linked lists and hash tables~\cite{Ashkiani:2018:ADH}, log-structured merge trees~\cite{Ashkiani:2018:GLA}, B-trees~\cite{Awad:2019:EAH}, quotient filters~\cite{GeilFO18}, and dynamic graphs~\cite{Awad:2020:DGO}. RXMesh's mesh data structure~\cite{Mahmoud:2021:RAG} was a result of this award. As well, the PIs recently designed and implemented the fastest GPU static hash tables~\cite{Awad:2023:AAI}.
%
\textbf{Broader Impacts.} Each of the above mentioned projects is available as open-source software, a significant reason why this work is highly cited. The dynamic graphs were integrated into Gunrock~\cite{Awad:2020:DGO,Wang:2017:GGG}, which subsequently was selected as the performance reference for the DARPA HIVE program and in turn integrated into NVIDIA's open-source RAPIDS framework for data science. Finally, NVIDIA's cuCollections library currently only supports static and dynamic maps but these were both influenced by the PIs' work and we expect as this library grows, it will incorporate the above mentioned work either directly or as the basis for NVIDIA's implementation and improvements.

PIs Bender and Farach-Colton's most relevant NSF-funded project ``CCF-BSF\@: AF\@: Small: Collaborative Research: The Dictionary Problem Considered''
(CCF-BSF 1716252, \$250,000, and CCF-BSF 1716252, \$250,000, 9/1/17--8/31/20).
The project is investigating algorithms inspired by systems requirements in dictionary data structures.
%\rob{The proposed research will investigate performance and functionality limitations of mmap that prevent its use in big data applications.}
\textbf{Intellectual Merit.} This grant has funded the following
publications~\cite{AgrawalBeDa20,AgrawalBeFi20,%
ArkinDaGa20,Ashkiani:2018:ADH,Ashkiani:2018:GLA,BenderChDa20,BenderCoFa19,BenderDaJo20,BenderFaGo18,%
BenderFaKu19,BenderGoMe20,BenderKoKu20,ChenMcSi18,ConwayBaJi17b,ConwayFaSh18,ConwayKnJi19,%
DasAgBe20,DasTsDu19,BerceaEv20a,BerceaEv20b,%
ZhanCoJi18,ZhanJaPo18,Mayer18,Pandey19,PandeyAlBe18,PandeyBJP17,%
PandeyBeJo18,Singh18,SinghMaBe20,%
JavanmardGaDa19DISC,%
GoswamiMeMe18,%
GeilFO18,pandey2020timely,%
EvenMeRa18,Awad:2019:EAH,BenderFiGi19,Farach-ColtonLiTs18,BenderKoPe18}.
%
\textbf{Broader Impacts.} The project has produced several software artifacts and involved several Ph.D.\ students, including two woman and a Hispanic student, as well as a woman undergraduate student.

For PI Patro, the most relevant NSF-frunded project influincing this proposal is ``CSR: Medium: Approximate Membership Query Data Structures in Computational Biology and Storage`` (for which PI Bender is a co-PI).  This project examines the use of AMQs, advanced hashing techniques, and related algorithms and data structures in the context of high-throughput genomics.
\textbf{Intellectual Merit.} This grant has funded, in whole or in part, the following publications~\cite{Khan2021,Khan2022,Khan2023CapsSA,Almodaresi2021,almodaresi2022incrementally,he2022alevin,Srivastava2019,fan2023fulgor,fan2023spt,skoufos2022agamemnon,singh2023treeterminus,He2023,Pibiri2023MacDBG,wu2023seesaw,Karami2023,He2023Ambiguity,Fan2022Perplexity,Mu2022Airpart,AlmodaresiPFJP19,Soneson2021,Srivastava2020AlnMap,Srivastava2020BayesianSharing,Sarkar2020Terminus,Love2020Tximeta,Marais2019Sketching,VandenBerge2019,Zhu2019,Zakeri2021LikeForLike,Sarkar2019Minnow,VanBuren2021}.
\textbf{Broader Impacts.} This grant has produced several open source tools, including \texttt{alevin-fry} which is becoming widely used in single-cell RNA-seq analysis, and the \texttt{cuttlefish} line of tools that are used in compacted de Bruijn graph construction, and has also contributed to extensive documentation and learning material for several related transcriptomics tools developed byt he lab. Research on this grant has involved several Ph.D.\ students and constituted the primary thesis work of a female Ph.D.\ student in PI Patro's group whose work earned the departmental dissertation award. 