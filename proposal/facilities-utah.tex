
\section*{University of Utah: Facilities, Equipment, and Other Resources}

The University of Utah provides several computing facilities for instructional and research use. The campus network backbone is a 40+Gbps network with a 100 Gbps research DMZ. The campus attaches via redundant 10 Gbps links to the Utah Education Network (UEN) which provides commodity internet and research connectivity. UEN maintains multiple gigabits of commodity from various carriers at strategic points throughout the state. For research connectivity, UEN connects via 10 Gbps directly with the 100 Gbps backbone of Internet2, both at the Salt Lake Level3 PoP. The University of Utah’s School of Computing attaches via redundant 10 Gbps connections to the campus backbone routed via OSPF; this provides desktop connections with 1 Gbps ethernet.

\subsection*{School of Computing}
The School of Computing’s computing infrastructure supplies many centralized services, including shared disk space (100 Terabytes), time, web/cgi/php, s/ftp, firewall, backups, printing resources, authentication (AD/LDAP/NIS), vpn, ssh/interactive servers, door lock access, and email. The core of the server infrastructure runs on VMware’s Enterprise virtualization products. Most services run on VM-Linux-based hosts, with some additional services being served from Windows machines. The School of Computing’s core instructional computing facility is the Computer Aided Design and Engineering (CADE) Lab. The CADE Lab includes approximately 75 personal computers running Centos 7.2 Linux with the 3.10 kernel, deployed on hardware equipped with an 3.60GHz Intel Core i7-4790 Processor, 32GB DDR3 1600MHz overclocked RAM, GeForce GTX 770/970 with 4GB of memory, and a 240GB Intel 540 Solid State Drive.

\subsection*{Center for High Performance Computing}
The University of Utah’s core research computing facility is the Center for High Performance Computing (CHPC), which supports the deployment and operation of large-scale and high performance computational resources, while also facilitating use of these resources through advanced user support and training. The CHPC also serves as an expert team to broadly support the diverse research computing needs on campus, in- cluding support for big data, big data movement, data analytics, security, virtual machines, Windows science application servers, protected environments for data mining and analysis of protected health information, and advanced networking. The CHPC manages over 22,000 cores and over 13 PB of RAID configured spinning disks. CHPC also leverages the national cyberinfrastructure for resources and training, including serving as partners in the ACI-REF (Advanced Cyberinfrastructure Research and Education Facilitators) program and as a member of RMACC (Rocky Mountain Advanced Computing Consortium).
