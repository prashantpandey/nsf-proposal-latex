%!TEX root =  proposal.tex

\section{Filters}

\prashant{Notes for Martin}

\prashant{Comp bio applications require Maplets on GPUs that support dynamic resizing and deletes.}

\prashant{Adaptive filters can be really useful for comp bio application as they can be used to construct MPH (minimal perfect hash) on GPUs.}

Filter data structures can be broadly classified into \emph{static} and
\emph{dynamic}.  Static filters approximately represent a set of items that must
be known before building the filter. Examples of these filters include XOR
filters~\cite{GrafLe20} and Ribbon filters~\cite{DillingerWalzer21}. Static
filters have recently seen more advancement due their use in storage
applications, such as LSM-trees.


In this paper, we consider dynamic filters as they have wide-spread applications
in data analytics.  Dynamic filters approximately represent a set of items that
does not need to be known before the construction. Dynamic filters have seen
much more advancement in the last few decades as applications often do not know
the set of items in advance. Examples of dynamic filters are Bloom
filters~\cite{Bloom70}, quotient filters~\cite{BenderFaJo12,
PandeyBJP17b,DillingerMa09,PaghPaRa05,EinzigerFr16}, and cuckoo
filters~\cite{FanAnKa14,BreslowJ18}.

\textbf{Bloom filters} consume $\log(e)\, \opt$ space, which is roughly
$\log(e)\approx 1.44$ times more than the lower bound of $\opt + \Omega(n)$
bits~\cite{CarterFG78}. In contrast, for a set $S$ taken from a universe $U$,
where $|U|=u$, an error-free dictionary requires $\Omega(\log {u\choose n})
\approx \Omega(n \log u)$ bits. Bloom filters also incur $\errbits$ cache-line
misses on inserts and positive queries, giving them poor insertion and query
performance.

\textbf{Blocked Bloom filters}~\cite{putze2007cache} overcome the poor cache
locality of Bloom filters by constructing a series of smaller Bloom filters each
of which is small enough to fit inside a small number of cache lines. The first
hash function is used to select a block and rest of the hash functions are used
to set/test bits inside the block. However, the cache efficiency comes at the
cost of higher false-positive rate. Blocked Bloom filters have theoretically and
empirically higher (up to $5\times$) false positive rates compared to Bloom
filters. See \Cref{tab:merged_fp_space} for the empirical calculations of FP
rate.

\textbf{Quotient
filters}~\cite{Cleary84,PaghPaRa05,DillingerMa09,BenderFaJo12a,PandeyBJP17,pandeySigmod21}
represent a set approximately by compactly storing small fingerprints of the
items in the set via Robin Hood hashing~\cite{CelisLaMu85}. The quotient filter
uses $1.053 (2.125 + \log_21/\epsilon)$ bits per element, which is less than the
Bloom filter whenever $\epsilon \leq 1/64$, which is the case in almost all
applications. It supports insertion, deletion, lookups, resizing, and merging.
The counting quotient filter (CQF)~\cite{PandeyBJP17}, improves upon the
performance of the quotient filter and adds variable-sized counters to count
items using asymptotically optimal space, even in large and skewed datasets. In
the counting quotient filter, we can also associate small values with items
either by re-purposing the variable-sized counters~\cite{PandeyABFJP18Cell} to
store values or by explicitly storing small values with the remainders in the
table~\cite{PandeySMB20}.

\textbf{Cuckoo filters}~\cite{FanAnKa14,BreslowJ18} also store small
fingerprints compactly in a table. However, unlike the quotient filter that uses
Robin Hood hashing, the cuckoo filter uses cuckoo hashing to resolve collisions
among fingerprints. Cuckoo hashing uses kicking (or cuckooing) to find an empty
slot for the new item when all the slots in a bucket are occupied. This results
in a cascading sequence of kicks until the filter converges on a new stable
state. Inserts become slower as the structure becomes full, and in fact inserts
may fail if the number of kicks during a single insert exceeds a specified
threshold (500 in the author's reference implementation).

\textbf{Two-Choice filters}~\cite{pandeySigmod21} organize fingerprints
compactly in blocks similar to the cuckoo filter. However, unlike the cuckoo
filter, there is no kicking. The blocks in the two-choice filter are larger in
size ($\approx \log{n}$, where $n$ is the number of items which is usually the
size of the cache line on most machines) than the cuckoo filter and
power-of-two-choice hashing is used to reduce the variance across the blocks and
achieve a high load factor. During insertions if both blocks corresponding to a
fingerprint are full then the data structure is declared full. The
power-of-two-choice hashing enables the filter to probe exactly two cache lines
during inserts and queries and write to a single cache line during inserts.
Given the larger block sizes the vector quotient filter~\cite{pandeySigmod21}
uses quotienting (similar to the quotient filter) to organize fingerprints
inside blocks. It divides the fingerprints into a quotient and remainder part
and only stores the remainder in the slot given by the quotient. It uses two
additional metadata bits to resolve collisions among quotients.


\subsection{Analysis of filter designs}

We now look at the dynamic filters discussed in~\Cref{sec:prelim} and evaluate
them based on the GPU design principles.  Our goal is to identify the filters
that offer necessary features such as deletions, counting, and value
associations and at the same time satisfy most of the design principles.  We
will further discuss the challenges of implementing these filter on the GPUs to
achieve high speed operations.

Bloom filters are easy to implement on the GPU as they only require test and set
operations. These operations can be implemented using atomic operations and
achieve low thread divergence. However, each operation results in multiple cache
misses and therefore Bloom filters have low memory coherence. They also have
sub-optimal space usage. Moreover, Bloom filters do not support deletions,
counting\footnote{The counting Bloom filter~\cite{FanCaAl00}, a variant of the
Bloom filter, supports counting but it comes at a high space-overhead which
makes it highly inefficient in practice.}, and associating small values with
items.
% that many data analytics applications require.

Blocked Bloom filters are better suited to GPUs.  Each operation requires
probing inside a single block. They achieve low thread divergence, high memory
coherence, a high degree of parallelism, and atomic operations. Thus, blocked
Bloom filters can satisfy all the GPU design principles. However, blocked Bloom
filters have a high false-positive rate compared to Bloom filters and also do
not support necessary features like deletions and counting.

Operations in the quotient filter have high cache locality which makes it an
appropriate choice to achieve high memory coherence. However, insert operations
in the quotient filter requires shifting fingerprints which makes it harder to
use atomic operations and also results in high thread divergence. However, the
quotient filter can support all the necessary features like deletions, counting,
and associating small values with items which makes the quotient filter a highly
usable data structure that multiple applications can benefit from.

It is quite challenging to achieve high speed operations while maintaining all
of the features in a GPU implementation of the quotient filter. Geil et
al.~\cite{Geil:2018:QFA} implemented a preliminary version of the GPU quotient filter.
However, that implementation was adapted from Bender et al.'s quotient
filter~\cite{BenderFaJo12a}, which did not have all the features, like counting
and value association, and also had higher space overhead. Furthermore, Geil et
al.'s GPU-based quotient filter has implementation-specific limitations (e.g., it
supports a fixed false-positive rate and can only be sized to store less than
$2^{26}$ items) resulting in poor performance and limited scalability.

The cuckoo filter stores fingerprints in fixed size blocks. This design is
amenable to high memory coherence and low thread divergence. Atomic operations
can also be used to read and write fingerprints. However, the cascading sequence
of reads and writes to random memory locations makes the cuckoo filter hard to
implement efficiently on the GPU\@. In particular, at high load factors when the
number of kicked items becomes high, each insertion will result in very low
memory coherence. Moreover, each kicking operation results in multiple
cache-line writes. This makes it challenging to achieve high speed operations in
a GPU cuckoo filter. Moreover, cuckoo filters do not support counting and
associating small values with items.

The two-choice filter has the advantages of the cuckoo filter design. It has
fixed size blocks. Each operation requires probing into exactly two blocks, and
inserts and deletes only write into a single block. This results in low thread
divergence, high memory coherence, and a high degree of parallelism. However,
due to large block sizes a more sophisticated structure is required to maintain
fingerprints inside each block. Therefore, it is not straightforward to use
atomic operations to read or write fingerprints inside blocks. It is a
challenging task to implement a two choice filter on the GPU using atomic
operations to achieve high throughput.

\section{Sketches}

\prashant{Notes for Martin.}

\prashant{Given a fastq file containing raw sequencing reads. Can we quickly and space-efficiently (maybe sublinear time and space?) estimate a frequency distribution for \kmers for a given $k$?
Specifically, we are interested in estimating the total number of \kmers, the total number of distinct \kmers, the total number of \kmers with a given count C?}

.....

\section{Hash tables}

\prashant{Notes for Martin/John.}

\prashant{Scale up: We need hash tables with: fine-grained resizing on GPUs. Applications often do not know the number of items in advance. If they under size the hash table they fail at run time and if they oversize they waste memory.}

\prashant{Scale out: We need hash table that can be distributed across multiple nodes in HPC clusters. However, when partitioning data across multiple nodes there are two challenges: First, we need to reduce the load variance across nodes. Second, we need to reduce the communication volume to while constructing the hash table and also to perform operations on the hash table. Operations can include: performing a join/intersection across two distributed hash tables.}

\prashant{Expand on the distributed HT usecase.}

Hash tables are a core data structure in many applications, including key-value
stores, databases, and big-data-analysis engines, and are included in most
standard libraries.  Hash-table performance can be a substantial bottleneck for
many applications~\cite{NealZu21,FanAn13,MetreveliZe12}.


We argue that two stricter criteria, \emph{referential stability} and \emph{low
associativity} should be optimized to yield high performance on PMEM.  As we
will see, these two goals seem to be at odds with each other, and part of the
innovation  of our hash table design is that it simultaneously achieves both.
Naturally, the third design goal for a high-performance hash table is
\emph{compactness}, but compactness also seems at odds with referential
stability and low associativity.

A hash table is said to be \defn{stable} if the position where an element is
stored is guaranteed not to change until either the element is deleted or the
table is resized~\cite{sandersstability,originalstability,KnuthVol3}.
Stability offers a number of desirable properties.  For example, stability
enables simpler concurrency-control mechanisms and thus reduces the performance
impact of locking.  Moreover, since elements are not moved, writing is
minimized, which improves PMEM performance.
expensive than reads~\cite{pmem-measurements}.

The \defn{associativity} of a hash table is the number of locations where an
element is allowed to be stored.\footnote{Associativity is often associated with
caches that restrict the locations an item may be stored in.  Here we refer to
\emph{data structural associativity}, which is a restriction on how many
locations a data structure may choose from to put an item in, even on fully
associative hardware.} The best known low-associative (DRAM) hash table is the
cuckoo hash table~\cite{Pagh:CuckooHash,PaghRo01}.  In the original design, each
element has exactly two locations in the table where it is allowed to be stored,
meaning that the associativity is two.  Low associativity yields a different set
of desirable properties---most importantly, it helps search costs. For example,
searching for an element in a cuckoo hash table is fast because there are only
two locations in the table to check.  In addition, low associativity can enable
us to further improve query performance by keeping a small amount of metadata;
see \Cref{sec:iceberght}.


In combination, stability can be used to achieve high insertion throughput in
PMEM, where writes are expensive, and low associativity can be use to achieve
high query  performance.  Furthermore, we also show how stability enables
locking and concurrency-control mechanisms to be simplified, leading to better
multithreaded scaling and simpler designs for crash consistency.

Unfortunately, there is a tension between stability and low associativity.  If a
hash table has associativity $\alpha$, and elements cannot move once they are
inserted, then an unlucky choice of $\alpha$ locations for $\alpha$ elements can
block a $(\alpha+1)$st element from being inserted.  As $\alpha$ decreases, the
probability of such an unlucky event increases.  Cuckoo hashing reduces the
probability of these bad events by giving up stability via \defn{kickout
chains}, which are chains of elements that displace each other from one location
to another. Practical implementations~\cite{LiAn14} generally increase the
number of elements that can be stored in a given location---and thus the
associativity---to reduce the kickout-chain length and increase the
maximum-allowed \defn{load factor}, i.e, the ratio of the total number of keys
in the table to the overall capacity of the table.


Similarly, there is a three-way tension between space efficiency, associativity,
and stability.  For example, cuckoo hash tables can be made stable if they are
overprovisioned so much that the kickout-chain length reaches 0.  Such
overprovisioning directly decreases space efficiency, but it also increases
associativity.  Linear probing hash tables are stable (assuming they use
tombstones to implement delete) but, as the load factor approaches 1, the
average probe length for queries goes up, increasing associativity.  Other
open-addressing hash tables have a similar space/associativity trade-off.
Chaining hash tables are stable, but they have large associativity and
significant space overheads.  CLHT~\cite{david2015asynchronized} improves query
performance despite high associativity by storing multiple items in each node,
but this further reduces space efficiency.

\iffalse
\subsection{B-trees}

The \btree(or B$^+$-tree\footnote{A \bplustree is a scan-optimized variant of
\btrees that stores all data records in the leaves and only pivot keys in the
internal nodes. The \bplustree is the widely implemented variant of the \btree
in real-world applications as it supports faster range
scans~\cite{mongodb,couchdb,scylladb,conway2020splinterdb,postgresql}. In this
paper, we refer to \bplustree as the \btree.})~\cite{BayerMc72} has been the
fundamental access path structure in databases and storage systems for over five
decades~\cite{Comer79,graefe2010survey}. \btrees are an extension of
self-balancing binary search trees to arbitrary fanouts (with more than two
children per node). They store elements in each node in a sorted array.  Given a
cache-line size $Z$~\cite{AggarwalVi88}, a \btree with $N$ elements and node
size $B = \Theta(Z)$ supports the point operations \proc{insert} and \proc{find}
in $O(\log_B(N))$ cache-line transfers in the I/O model~\cite{AggarwalVi88}.
\btrees are one of the top choices for in-memory indexing~\cite{ZhangChOo15} due
to their cache efficiency though they were initially introduced for indexing
data stored on disk~\cite{BayerMc72}. In this paper, we study and improve the
performance of in-memory \btrees.

\btrees are especially popular in databases and file systems because they
support logarithmic point operations (inserts and finds) and efficient range
operations (range queries and scans) that read blocks of
data~\cite{Knuth98,rodeh2013btrfs}.  They are also extensively used as the
in-memory index in many popular databases such as MongoDB~\cite{mongodb},
CouchDB~\cite{couchdb}, ScyllaDB~\cite{scylladb}, PostgreSQL~\cite{postgresql},
and SplinterDB~\cite{conway2020splinterdb}.
\fi

\section{String data structures}





