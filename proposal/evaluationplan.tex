\section{Evaluation plan}

% The Project Description must contain a section titled "Evaluation Plan" that (a) includes a timeline of proof-of-concept implementations of the key components; and (b) outlines how the project's success will be measured.

\subsection{Timeline}
\begin{description}
\item [Year 1:] Develop theoretical models for GPU-based data structures. Build initial prototypes of filters, hash tables, sketches, and string data structures for single GPUs. [Task 1, Task 4]

\item [Year 2:] Explore computational biology applications such as taxonomic classification for metagenomic data, raw sequence search, and pangenomics to develop a common data structure API. Build single GPU-enabled software tools for common data processing pipelines such as k-mer analysis and sequence alignment. [Task 5, Task 10]

\item [Year 3:] Develop theoretical models for single-node-multi GPU data structures. Develop a single-node-multi-GPU compact and dynamic data structure library with strong theoretical guarantees. Integrate the data structure library in computational biology tools. [Task 2, Task 6, Task 11]

\item [Year 4:] Develop theoretical models for multi-node multi-GPU data structures for computational biology problems. Build prototype multi-node multi-GPU data structures. [Task 3, Task 7]

\item [Year 5:] Scale out computational biology applications such as taxonomic classification for metagenomic data, raw sequence search, and pangenomics in HPC environments such as Perlmutter and Summit supercomputers using the multi-node multi-GPU data structures library and perform petabyte scale data analyses. [Task 8, Task 9, Task 12, Task 13]
\end{description}

\begin{ganttchart}[
  x unit=0.6cm,
  vgrid,
  hgrid,
  y unit title=0.4cm,
  title height=1,
  y unit chart=0.4cm,
  bar height=0.6,
  ]{1}{20}
  \gantttitle{2023}{1}
  \gantttitle{2024}{4}
  \gantttitle{2025}{4}
  \gantttitle{2026}{4}
  \gantttitle{2027}{4}
  \gantttitle{2028}{3}\\
  %%%%%%%%%%%%%%%%%%%%%%%%%%%%
  \ganttgroup{Algorithms}{1}{20}\\
  \ganttbar{Task 7}{1}{3}\\
  \ganttbar{Task 8}{4}{7}\\
  \ganttbar{Task 9}{7}{10}\\
  \ganttbar{Task 11}{11}{14}\\
  \ganttbar{Task 17}{4}{7}\\
  %%%%%%%%%%%%%%%%%%%%%%%%%%%%
  \ganttgroup{Systems}{1}{20}\\
  \ganttbar{Task 5}{1}{3}\\
  \ganttbar{Task 6}{3}{5}\\
  \ganttbar{Task 10}{6}{9}\\
  \ganttbar{Task 13}{10}{2}\\
  \ganttbar{Task 14}{3}{6}\\
  \ganttbar{Task 15}{7}{10}\\
  \ganttbar{Task 18}{8}{12}\\
  \ganttbar{Task 19}{13}{16}\\
  %%%%%%%%%%%%%%%%%%%%%%%%%%%%
  \ganttgroup{HPC}{1}{20}\\  
  \ganttbar{Task 12}{11}{14}\\
  \ganttbar{Task 16}{13}{16}\\
  \ganttbar{Task 20}{17}{20}\\
  %%%%%%%%%%%%%%%%%%%%%%%%%%%%
  \ganttgroup{Applications}{1}{20}\\  
  \ganttbar{Task 1}{3}{16}\\
  \ganttbar{Task 2}{5}{18}\\
  \ganttbar{Task 3}{2}{20}\\
  \ganttbar{Task 4}{3}{20}
\end{ganttchart}

\subsection{Measuring Success} We will combine our systems and theory advances into a standalone GPU library for compact, dynamic, and distributed data structures that we will link with our focus computational biology applications. We expect to be measured on the success of these the data-structure library and these applications.

\noindent
\paragraph{Success for our data-structure library}

\label{sec:sw-methodology}

\begin{itemize}[noitemsep]
  \item Our library is the data-structure substrate for the three computational biology applications described below, for both single-GPU and multi-GPU/multi-node configurations.
  \item Our library embodies best practices in software development: open-source software with an Apache 2.0 license, available via Github; a suite of functional and performance tests; a benchmarking suite to evaluate the performance of these data structures and help drive subsequent modules; use of code-coverage tools together with these tests; use of continuous-integration and continuous-development tools and methodologies; internally documented code with Doxygen; and high-level developer documentation that guides use of the library. Particularly with our widely cited and used Gunrock library~\cite{Wang:2017:GGG} for GPU graph analytics, we have extensive experience with building and deploying high-quality academic software.
  \item We quantitatively measure the success of our library to others with github download and star counts; with issues filed and closed; with the number of pull requests from external developers; and primarily, via the use of our library in other projects, where we will measure both uses by developers with whom we develop as well as uses from those with whom we do not.
\end{itemize}

\paragraph{Success of computational biology applications}

For each of our three applications (taxonomic classification, raw sequence search, and pangenomics), success means:

\begin{itemize}[noitemsep]
  \item We will scale each of these applications to process and perform complex data analyses on biggest datasets available. For example, we will scale taxonomic classification on terabyte-scale metagenomic datasets from WA, Rhizo, and Tymeflies~\cite{hofmeyr2020terabase}; index and query raw sequence search in real time on all of SRA at petabyte scale~\cite{kodama2012sequence}; build a pangenomic index at population scale (100,000 genomes project~\cite{1002021100}) and perform fast variant queries and sequence-to-graph alignment.
  \item For all of these applications we will build open-source software tools with an Apache 2.0 license, available via Github. Particularly with our widely cited and used computational biology tools~\cite{PandeyABFJP18Cell,PandeyBJP17a,PandeyBJP17b,PandeyBJP17,pandey2020timely,pandey2021variantstore,pandey2021terrace,pandey2022iceberght}, we have extensive experience with building and deploying high-quality academic software.
  \item Best-in-class performance (vs.\ CPUs) on a single node.
  \item Achieved strong and weak scalability across multiple GPUs and multiple nodes in an HPC environment. 
  \item Best-in-class performance (vs.\ CPUs) on multiple nodes.
\end{itemize}
