\section{Evaluation plan}

\subsection{Datasets}

\paragraph{Raw sequencing data}
Over the past decade the cost of sequencing has decreased dramatically, making the generation of sequence data more accessible. This has led to increasingly ambitious sequencing projects. For example, the 1,000 Genomes Project, which began in
2008 and completed in 2012 (Clarke et al., 2012), led to the 100,000 Genomes Project, which began in 2014 and completed
in 2018 (Turnbull et al., 2018). There are dozens of other large-scale sequencing projects completed or underway, including
GEUVADIS (Lappalainen et al., 2013), GenomeTrakr (Timme et al., 2018), and MetaSub (The MetaSUB International Consortium, 2016). An overwhelming amount of public data is now available at EBI’s European Nucleotide Archive (ENA) (Cook
et al., 2018) and NCBI’s Sequence Read Archive (SRA) (Leinonen et al., 2010). The possibility of analyzing these collections
of datasets, alone or in combination, creates vast opportunities for scientific discovery, exceeding the capabilities of traditional
laboratory experiments.

\john{This isn't an ``evaluation plan''. This is background.}

\paragraph{Metagenomic data}

\paragraph{Pangenomic data}
