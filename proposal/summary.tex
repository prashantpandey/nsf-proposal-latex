%!TEX root =  proposal.tex

%% Don't actually need this - will submit txt.  Just want to estimate actual space
\begin{center}
\bf
\Large
Collaborative Research: PPoSS: Large: A Compact, Dynamic, and Distributed Data
Structure Library for Computational Biology

\medskip
\small
Michael A. Bender (SBU), Martin Farach-Colton (Rutgers), John Owens (UC Davis),
Prashant Pandey (Utah)
\end{center}


\noindent \textbf{\large Overview:}
Raw sequencing data in computational biology is increasing faster than ever due to high-throughput sequencing technology (HTS), which is already producing petabyte-scale datasets. Many applications in computational biology (\kmer analysis, raw sequence search, taxonomic classification, and pangenomics) require processing raw sequencing data at petabyte scales. This project aims to build high-performance and scalable data analysis-pipelines for computational-biology applications. Specifically, we aim to develop new massively-parallel and distributed data structures and algorithms for core computational biology data processing tasks. The new data structures and algorithms that will have wide applicability in computation biology applications and beyond. 

We will develop new \textbf{algorithmic theory} to design scalable data structures; we will develop new \textbf{systems} that implement our solutions in a scale-up manner,  both on CPU as a prototype and GPUs for massive-parallelism; we will develop a new framework to distribute our solutions in an \textbf{HPC} environment, so they  scale out to clusters of GPUs; and finally we will validate out solutions on \textbf{computational biology} workloads.
%
Specifically, our data structures are: \textbf{filters}, \textbf{hash tables}, \textbf{string indexes}, \textbf{sketches}, and related auxialiary data structures.  We will validate our solutions on \textbf{large-scale raw sequence search}, \textbf{taxonomic classification for metagenomic data}, \textbf{pangenomic indexing and analysis}, and others.


\noindent \textbf{\large Keywords:} Data structures, algorithms, computational biology, GPUs, high-performance computing.

\noindent \textbf{\large Intellectual Merit:}
If successful we will be able to perform complex data analyses to answer biological questions on terabyte and petabyte scale datasets. For example, raw sequencing data from SRA is already at petabyte scale, metagenomic data from Western Arctic and Rhizo are in terabytes, and population-scale pangenomic data from 100,000 genome project. Existing data structures and software tools fail to scale to these data sizes making these publicly-available and highly-valuable data resources largely inert.

The project’s novelties are: a vertical-stack approach spanning theory and algorithms: highly-concurrent, dynamic, and distributed data structures, systems: scale up using GPU acceleration, high-performance computing: scale out using distributed data structures, and applications: computational biology applications; new parallel and distributed data structures and algorithms to exploit the massive compute on GPUs applicable to other application domains; an API for developers to quickly and seamlessly integrate high-performance and scalable data structures in applications.
%
Our team includes a highly interdisciplinary team of researchers across four focus areas: applications (computational biology), theory and algorithms, systems, and high-performance computing. The team is taking a holistic theory/systems/HPC/applications co-design approach to explore four tightly interconnected research modules. These research modules are structured from bottom-up across the computing stack.

\noindent \textbf{\large Broader Impacts: }\\
The primary broader impact of this work is to enhance the capability of bioinformatics applications to perform computations at the largest scales. Both accelerated computation (allowing quicker feedback and more experiments) and larger computation potentially accelerate the process of scientific discovery. Secondly, we expect that a high-quality library of distributed, scalable GPU data structures will find widespread utility in other application domains.

Beyond technical broader impacts, we propose outreach impacts through tutorial on using our data structure library in computational biology conferences and proposing week-long research seminars like Dagstuhl that we hope will bridge the gaps between computational biologists, CS theorists, and CS systems researchers.
