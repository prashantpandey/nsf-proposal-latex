% -*- Mode: Latex -*-
%\documentclass[11pt,letterpaper]{article} 
% It turns out we can use 10-point if use palatino

\input{header}
\pagenumbering{gobble}



\begin{document} 

\thispagestyle{empty}

\subsection*{Collaboration Plan}

The PIs on this project have a long history of collaboration—across many areas of algorithms, computational biology, HPC, algorithms system building, application development, and technology transfer. For example, they have written over 90 papers together in pairs, triples, and more, have started a company together, have obtained numerous grants together, have taught tutorials all over the world together, have co-advised each other’s students, have hosted each others postdocs, etc.

Following our successful pattern from past research projects of both in person and virtual collaboration, we will have weekly Zoom meetings and (at least) two in-person retreats per year. 

\paragraph{Team.} The work proposed in this project requires communication across multiple areas of CS, mathematics, and software engineering that often do not talk to each other.

Specifically, vector search indexing brings together graph algorithms, data structures, and randomization. Vector compression draws from sketching, data structures, and randomization. As systems scale
beyond the memory and computational capabilities of single server, these techniques need to extend to
multicore, GPU and distributed settings, or leverage external storage. To do so requires expertise in parallel algorithms, algorithms for GPUs, and distributed algorithms, as well as external-memory algorithms.
Furthermore, developing relevant theory in this area requires experience with AI and ML applications and
at-scale vector search systems, and more generally experience with tech transfer from theory to practice.
Our team consists of researchers that bring together these essential qualifications.

Team members with special expertise will will lead particular thrusts, although all team members will collaborate on all aspects of this project.
\begin{itemize}%[noitemsep,nolistsep]
    \item \textbf{Theory and Algorithms}: PIs Bender, Farach-Colton, Pandey, and Patro
    

    \item \textbf{GPU Systems}: PIs Farach-Colton, Owens, and Pandey

    \item \textbf{High-Performance Computing}: PIs Bender, Farach-Colton, Owens, and Pandey

    \item \textbf{Large-scale computational biology} PIs Bender, Farach-Colton, Pandey, and Patro

\end{itemize}



\paragraph{A record of collaboration.} The team has (in pairs or triples) published over 90 papers together, filed for
9 patents, and started a company, Tokutek, to commercialize their collaborative work. We have worked
together to create and manage a large piece of software resulting in multiple award-winning papers and
outreach tutorials. In total, we have a team with precisely the right mixture of GPU, HPC, algorithms, and architecture expertise to ensure the success of this project.

The success of the PIs’ past interaction is due, in no small part, to the complementary skills of the PIs,
spaning a wide range of areas including the theoretical foundations of AI, multiple areas of algorithms, and
system building. The cross-pollination from disparate areas of
computer science has led to the deployment of sophisticated algorithms in large-scale systems.

\paragraph{Joint advising/mentoring.} The PIs have a long history of joint mentoring and advising.  Additionally, the PIs regularly serve as mentors/joint advisors/coauthors/committee members/hosts for each other’s students and postdocs.


\end{document}