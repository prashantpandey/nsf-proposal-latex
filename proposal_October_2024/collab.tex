% -*- Mode: Latex -*-
%\documentclass[11pt,letterpaper]{article} 
% It turns out we can use 10-point if use palatino


\thispagestyle{empty}

\subsection*{Collaboration Plan}

The PIs on this project have a long history of collaboration—across many areas of systems, algorithms, computational biology, HPC, application development, and technology transfer. For example, they have written over 90 papers together in pairs, triples, and more, have started a company together, have obtained numerous grants together, have taught tutorials all over the world together, have co-advised each other’s students, have hosted each others postdocs, etc.

Following our successful pattern from past research projects of both in person and virtual collaboration, we will have weekly Zoom meetings and (at least) two in-person retreats per year. 

\paragraph{Team.} The work proposed in this project requires communication across multiple areas of CS and software engineering that often do not talk to each other.

% Specifically, vector search indexing brings together graph algorithms, data structures, and randomization. Vector compression draws from sketching, data structures, and randomization. As systems scale
% beyond the memory and computational capabilities of single server, these techniques need to extend to
% multicore, GPU and distributed settings, or leverage external storage. To do so requires expertise in parallel algorithms, algorithms for GPUs, and distributed algorithms, as well as external-memory algorithms.
% Furthermore, developing relevant theory in this area requires experience with AI and ML applications and
% at-scale vector search systems, and more generally experience with tech transfer from theory to practice.
% Our team consists of researchers that bring together these essential qualifications.

All team members will collaborate on all aspects of this project.

\begin{description}%[noitemsep,nolistsep]

    \item[GPU Systems (PIs Farach-Colton, Owens, and Pandey)] Farach-Colton, Owens, and Pandey have written numerous paper on GPU data structures and programming models. Owens' research program in GPU computing~\cite{Owens:2007:ASO,Owens:2008:GC} spans nearly 20 years and includes representative research advances in fundamental algorithms~\cite{Sengupta:2007:SPF}, data structures~\cite{Lefohn:2006:GGE,Alcantara:2009:RPH}, % scalability to multiple GPUs~\cite{Stuart:2009:MPO,Stuart:2011:EMT,Stuart:2011:MMO,Pan:2017:MGA,Pan:2018:SBS,Chen:2022:SIP},
    performance engineering~\cite{Zhang:2011:AQP}, programming models~\cite{Gupta:2012:ASO, Tzeng:2010:TMF}, and applications~\cite{Wang:2017:GGG}. Pandey's research includes building massively parallel and feature-rich GPU filters~\cite{mccoy2022high}, distributed-memory GPU hash tables for efficiently processing genomic data~\cite{nisa2021distributed}, GPU-accelerated exascale metagenomic assembly pipelines~\cite{McCoyHYP23a}, and high-performance and general-purpose GPU memory managers~\cite{McCoyP24}.

    \item[Design and Analysis of Data Structures (PIs Bender, Farach-Colton, Pandey, and Patro)]
    Bender, Farach-Colton, Pandey, and Patro have written numerous theory related publications and have made significant contributions towards designing efficient data structures and building scalable applications using their theoretical contributions~\cite{BenderFaGo18,BenderFaJo12,PandeyBJ17,PandeyAlBe18,PandeyBJP17,PandeyBeJo17b,ConwayFaSh18,JannenYuZh15a,JannenYuZh15b,YuanZhJa16,pandey2021terrace,pandey2021variantstore,PandeyBeCo23,Assadi2023,Bender2023,pandey2020timely,PandeyCDBFJ21,mccoy2022high,Almodaresi2018Pufferfish,fan2023spt,Khan2021,Khan2022,Khan2023CapsSA,fan2023fulgor,Pibiri2023MacDBG}.


    \item[High-Performance Computing (PIs Bender, Farach-Colton, Owens, and
        Pandey)] Bender has
      written a number of top-tier related papers in HPC~\cite{pandey2020timely,bender2017two,eckstein2015pebbl,agrawal1989four,bender2008communication,greenberg1999enabling},
      and had considerable impact on HPC practice.
      PI Bender's work in HPC has focused on scheduling and  won a joint R\&D 100 Award for processor scheduling and allocation algorithms, which were licensed by Cray and incorporated into SLURM\@.  PIs Bender and Farach-Colton's company Tokutek deployed software to manage metadata in a large cloud storage service. Owens led the first implementation of MPI on GPUs~\cite{Stuart:2009:MPO:withouturl,Stuart:2011:EMT}, the first multi-GPU MapReduce~\cite{Stuart:2011:MMO}, and more recent work on scalable graph analytics on HPC machines~\cite{Pan:2018:SBS,Pan:2017:MGA,Chen:2022:SIP}. Pandey has built the first GPU-based distributed-memory \kmer analysis pipeline for the MetaHipMer metagenome assembler~\cite{nisa2021distributed}.

    \item[Large-scale computational biology (PIs Bender, Farach-Colton, Pandey, and Patro)] Pandey and Patro's work in computational biology has focused on rebuilding key parts of genomic, transcriptomic, and pangenomics analysis tool-chains around data structures and algorithms of their design, and have been having tremendous impact in the field. This work is described in flagship computational biology conferences (RECOMB, ISMB, WABI) and journals (Cell Systems, Genome Biology, Bioinformatics, JCB, Nature Biotechnology and Nature Methods)~\cite{PandeyAlBe18,PandeyBJP17,PandeyBeJo17b,AlmodaresiPFJP20,pandey2021variantstore,almodaresi2017rainbowfish,almodaresi2022incrementally,PatroSailfish:2014,Patro2017Salmon,Srivastava2019,he2022alevin,Almodaresi2018Pufferfish,Almodaresi2021}.  PI Farach-Colton has papers in genome assembly~\cite{Choi2003}, phylogeny construction~\cite{Farach97,Ambainis97,FarachKKM97,Farach1999, Cohen1997}, and string indexes~\cite{Farach97,Ambainis97}.

\end{description}




\paragraph{A record of collaboration.} The team has (in pairs or triples) published over 90 papers together, filed for
9 patents, and started a company, Tokutek, to commercialize their collaborative work. We have worked
together to create and manage a large piece of software resulting in multiple award-winning papers and
outreach tutorials. In total, we have a team with precisely the right mixture of GPU, HPC, algorithms, and architecture expertise to ensure the success of this project.

The success of the PIs’ past interaction is due, in no small part, to the complementary skills of the PIs,
spanning a wide range of areas including the theoretical foundations of AI, multiple areas of algorithms, and
system building. The cross-pollination from disparate areas of
computer science has led to the deployment of sophisticated algorithms in large-scale systems.

\paragraph{Joint advising/mentoring.} The PIs have a long history of joint mentoring and advising.  Additionally, the PIs regularly serve as mentors/joint advisors/coauthors/committee members/hosts for each other’s students and postdocs.
