% -*- Mode: Latex -*-
%\documentclass[11pt,letterpaper]{article} 
% It turns out we can use 10-point if use palatino

%!TEX root =  proposal.tex

\usepackage[T1]{fontenc}
%\usepackage[latin1]{inputenc}
\usepackage[amssymb]{SIunits}
\usepackage{latexsym}
\usepackage{multirow}
\usepackage{wrapfig}
\usepackage{amsmath, amsthm, amssymb}
\usepackage{amsfonts}
\usepackage[format=plain,indention=0cm, font=small, labelfont=bf]{caption}
\usepackage{longtable}
\usepackage{fancyhdr}
\usepackage[pdftex]{graphicx}
\usepackage[pdftex,
	    colorlinks,
	    pdfstartview=FitH,
	    linkcolor=black,
	    citecolor=black,
	    urlcolor=black,
	    filecolor=black
	    ]{hyperref}
\usepackage{lscape}
\usepackage{floatrow}
%\usepackage{enumerate}
%\usepackage{enumitem}
\usepackage{tabularx}
\usepackage{ragged2e}

\usepackage[normalem]{ulem} % allows text to be striked out. use \sout{xxx}

% \definecolor{coolblack}{rgb}{0.0, 0.18, 0.39}
% \definecolor{darkcerulean}{rgb}{0.03, 0.27, 0.49}
% 
% \newcommand{\propose}[1]{{\color{darkcerulean}#1}}

\usepackage[table]{xcolor}
\usepackage{comment}
\usepackage{ifthen}

\newcolumntype{Y}{ >{\RaggedRight\arraybackslash}X}
\newcommand\T{\rule{0pt}{2.6ex}}
\newcommand\B{\rule[-1.2ex]{0pt}{0pt}}

\renewcommand{\refname}{References Cited}

\newcommand{\degrees}{$\!\!$\char23$\!$}

\def\rrr#1\\{\par
\medskip\hbox{\vbox{\parindent=2em\hsize=6.12in
\hangindent=4em\hangafter=1#1}}}

\renewcommand\baselinestretch{1}
%%% \pagestyle{empty}
%%% INCLUDE THE ABOVE TO PUT PAGE NUMBERS BACK IN

%\setlength{\parindent}{0pt}
\setlength{\parskip}{1ex}

%!TEX root =  proposal.tex

\usepackage[moderate,bibnotes,charwidths,mathdisplays,indent]{savetrees}

%\usepackage[textsize=tiny]{todonotes} % package for cool todonotes in the margin
%\usepackage[normalem]{ulem} % package that allows text to be striked out
\usepackage{titlesec}

\titlespacing*{\section}
{0pt}{2.5ex plus 1ex minus .2ex}{1.3ex plus .2ex}
\titlespacing*{\subsection}
{0pt}{2.5ex plus 1ex minus .2ex}{1.3ex plus .2ex}


\usepackage{amsmath}
\usepackage{amsthm}
\usepackage{amstext}
\usepackage{amsfonts}
\usepackage{mathrsfs}
\usepackage{xcolor}
\usepackage{tikz}
\usepackage{cleveref}
\usepackage{subcaption}
\usepackage{cite}
\usepackage{pgfplots}
\pgfplotsset{compat=1.3}
\usepackage{url}
\urlstyle{tt}
\usepackage{xspace}
%\usepackage{algorithm}
%\usepackage[noend]{algpseudocode}
%\usepackage{siunitx}
\usepackage{booktabs}
\usepackage{multirow}
\usepackage{comment}
\usepackage[title]{appendix}
\usetikzlibrary{external}
%\usepackage{enumitem}
\usepackage{balance}
\usepackage{pgfgantt}

\iffalse
\usepackage{xr}
% In your preamble

\makeatletter
\newcommand*{\addFileDependency}[1]{% argument=file name and extension
  \typeout{(#1)}
  \@addtofilelist{#1}
  \IfFileExists{#1}{}{\typeout{No file #1.}}
}
\makeatother

\newcommand*{\myexternaldocument}[1]{%
    \externaldocument{#1}%
    \addFileDependency{#1.tex}%
    \addFileDependency{#1.aux}%
}
\myexternaldocument{supplementary}
\fi



\usepackage{graphicx}

\usepackage{svg}

%\usepackage{amssymb}

%!TEX root =  proposal.tex


%% MAB: for proposing text
\definecolor{coolblack}{rgb}{0.0, 0.18, 0.39}
\definecolor{darkcerulean}{rgb}{0.03, 0.27, 0.49}
\newcommand{\propose}[1]{{\color{darkcerulean}#1}}

% MAB: this is the macro we can use to enumerate research problems
\newtheorem{eval}{Evalulation}
\newtheorem{prob}[eval]{Research Problem}
\newtheorem{rproblem}[eval]{Research Problem}
\newtheorem{proofofconcept}[eval]{Proof of Concept}

\newcommand{\evalref}[1]{Evaluation~\ref{#1}}
\newcommand{\evalreftwo}[2]{Evaluations \ref{#1} and~\ref{#2}}
\newcommand{\evalrefthree}[3]{Evaluations \ref{#1}, \ref{#2},             and~\ref{#3}}
\newcommand{\evalreffour}[4]{Evaluations \ref{#1}, \ref{#2}, \ref{#3},                   and~\ref{#4}}


\newcommand{\errbits}{\log (1/\epsilon)}
\newcommand{\opt}{n\errbits}
\renewcommand{\epsilon}{\varepsilon}

% \newcommand{\defn}[1]{\textit{#1}}
\newcommand{\defn}[1]{{\textit{\textbf{\boldmath #1}}}\xspace}
\newtheorem{theorem}{Theorem}
\newtheorem{lemma}[theorem]{Lemma}
\newtheorem{corollary}[theorem]{Corollary}
\newtheorem{claim}[theorem]{Claim}
\newtheorem{proposition}[theorem]{Proposition}
\newtheorem{assumption}[theorem]{Assumption}
\newtheorem{definition}[theorem]{Definition}
\newtheorem{invariant}[theorem]{Invariant}
\newtheorem{observation}[theorem]{Observation}

\ifdefined\textln\relax\else\newcommand{\textln}[1]{#1}\fi

%\makeatletter
%\algrenewcommand\ALG@beginalgorithmic{\footnotesize}
%\makeatother


\definecolor{magenta4}{rgb}{0.5625,0,0.5625}
\definecolor{green4}{rgb}{0,0.5625,0}
\definecolor{orange4}{rgb}{0.98,0.31,0.09}
\definecolor{powderblue}{rgb}{0.69,0.88,0.9}

\iftrue
\newcommand{\mab}[1]{{\scriptsize \textcolor{blue}{Michael: {#1}}}}
\newcommand{\mfc}[1]{{\scriptsize \textcolor{orange4}{Martin: {#1}}}}
\newcommand{\john}[1]{{\scriptsize \textcolor{pink}{John: {#1}}}}
\newcommand{\prashant}[1]{{\scriptsize \textcolor{cyan}{Prashant: {#1}}}}
\fi


% see http://latexcolor.com to define more colors

\newcommand{\fixme}[1]{\todo[size=\tiny]{#1}}

\newcommand{\TODO}{{\color{red}\bf TODO}}


\newcommand{\qf}{quotient filter\xspace}
\newcommand{\cqf}{counting quotient filter\xspace}
\newcommand{\bloom}{Bloom filter\xspace}
\newcommand{\btree}{B-tree\xspace}
\newcommand{\btrees}{B-trees\xspace}
\newcommand{\bplustree}{B$^+$-tree\xspace}
\newcommand{\bplustrees}{B$^+$-trees\xspace}
\newcommand{\proc}{\texttt}


\newcommand{\kmer}{$k$-mer\xspace}
\newcommand{\Kmer}{$K$-mer\xspace}
\newcommand{\kmers}{$k$-mers\xspace}
\newcommand{\Kmers}{$K$-mers\xspace}


\newcommand{\poly}[1]{\textrm{poly}{(#1)}}
\newcommand{\polylog}[1]{\textrm{polylog}{(#1)}}

\newcommand{\para}[1]{\smallskip\noindent\textbf{#1.}}

\newcommand{\punt}[1]{}

\pagenumbering{gobble}



\begin{document} 

\thispagestyle{empty}

\subsection*{Collaboration Plan}

\mfc{This is gotten from another proposal and I haven't finished updating it.}


The PIs on this project have a long history of collaboration—across many areas of algorithms, computational biology, HPC, algorithms system building, application development, and technology transfer. For example, they have written over 90 papers together in pairs, triples, and more, have started a company together, have obtained numerous grants together, have taught tutorials all over the world together, have co-advised each other’s students, have hosted each others postdocs, etc.

Following our successful pattern from past research projects of both in person and virtual collaboration, we will have weekly Zoom meetings and (at least) two in-person retreats per year. 

\paragraph{Team.} The work proposed in this project requires communication across multiple areas of CS, mathematics, and software engineering that often do not talk to each other.

Specifically, vector search indexing brings together graph algorithms, data structures, and randomization. Vector compression draws from sketching, data structures, and randomization. As systems scale
beyond the memory and computational capabilities of single server, these techniques need to extend to
multicore, GPU and distributed settings, or leverage external storage. To do so requires expertise in parallel algorithms, algorithms for GPUs, and distributed algorithms, as well as external-memory algorithms.
Furthermore, developing relevant theory in this area requires experience with AI and ML applications and
at-scale vector search systems, and more generally experience with tech transfer from theory to practice.
Our team consists of researchers that bring together these essential qualifications.

Team members with special expertise will will lead particular thrusts, although all team members will collaborate on all aspects of this project.
\begin{itemize}
\item Algorithms for ML and data science (PI Musco)
\item At-scale vector search systems (PI Dhulipala)
\item Sketching for matrix and vector data (PIs Farach-Colton and Musco)
\item Streaming algorithms (PIs Farach-Colton and Musco)
\item Parallel algorithms and data structures. (PIs Bender, Conway, Dhulipala, Farach-Colton)
\item External memory algorithms and data structures (PIs Bender, Conway, Farach-Colton)
\item GPU data structures (PI Farach-Colton)
\item Tech transfer from theory to systems (PIs Bender, Conway, Dhulipala, Farach-Colton)
\end{itemize}

\paragraph{A record of collaboration.} The team has (in pairs or triples) published over 90 papers together, filed for
9 patents, and started a company, Tokutek, to commercialize their collaborative work. We have worked
together to create and manage a large piece of software resulting in multiple award-winning papers and
outreach tutorials. In total, we have a team with precisely the right mixture of foundational AI, foundational
theory, algorithms, OS, and architecture expertise to ensure the success of this project.

The success of the PIs’ past interaction is due, in no small part, to the complementary skills of the PIs,
spaning a wide range of areas including the theoretical foundations of AI, multiple areas of algorithms, and
system building (both for AI systems and non-AI systems). The cross-pollination from disparate areas of
computer science has led to the deployment of sophisticated algorithms in AI systems and other large-scale
systems.

\paragraph{Joint advising/mentoring.} The PIs have a long history of joint mentoring and advising. PI Dhulipala has
two current PhD students who were previously undergraduate research advisees of PIs Bender and Farach-
Colton. PI Conway has a current PhD student who was previously an undergraduate research advisee of PI
Dhulipala. PIs Farach-Colton and Musco currently run a joint group meeting at NYU. Additionally, the PIs
regularly serve as mentors/joint advisors/coauthors/committee members/hosts for each other’s students and
postdocs.


\end{document}