%!TEX root =  proposal.tex

%% Don't actually need this - will submit txt.  Just want to estimate actual space
\begin{center}

\bf
\Large
Collaborative Research: CSR: Medium: \\Dynamic and Distributed Data Structures on the GPU

% \medskip
\small
Lead PI: Mart\'{i}n Farach-Colton (NYU) \\
Co-PIs: Michael A. Bender (SBU), John Owens (UC Davis), Prashant Pandey (Northeastern), Rob Patro (UMD)
\end{center}

\vspace{-0.3cm}

\noindent \textbf{\large Overview:}\\
Both the volume and variety of data has been increasing at an ever faster rate
driven by our ever increasing ability to generate data across fields likes
biology, astrophysics, quantum chemistry, etc. Applications across these
fields require processing data at petabyte scales. These applications use a set
of common data structures to perform many data-analysis tasks, and their
performance is bottlenecked by the data-structure and algorithm performance.

This project aims to develop high-performance and scalable data
analysis-pipelines for modern data-intensive applications. Specifically, we aim
to develop new massively-parallel and distributed data structures and
algorithms that sit at the heart of many data processing tasks. These new data
structures and algorithms will have wide applicability across applications.
%
We will develop new \defn{algorithmic theory} to design scalable data
structures and new \defn{systems} that implement our solutions in a scale-up
manner on GPUs. We will develop a new framework to distribute our solutions
across nodes, so they scale out to clusters of GPUs; and finally we will
validate our solutions on \defn{computational biology} workloads.
%
Specifically, the data structures we propose to study are: \defn{filters},
\defn{sketches}, \defn{hash tables}, \defn{string indexes}, and related
data structures.  We will validate our solutions on \defn{large-scale raw
sequence search}, \defn{single-cell preprocessing}, \defn{taxonomic
classification for metagenomic data},  and others.

\noindent \textbf{\large Keywords:} Data structures, algorithms, computational
biology, GPUs, high-performance computing (HPC).

\noindent \textbf{\large Intellectual Merit:}\\
If successful we will be able to quickly perform complex data analyses on
terabyte- and petabyte-scale datasets. For example, raw
sequencing data from sequence read archive (SRA) is already at petabyte scale,
metagenomic data from Western Arctic and Rhizo are in terabytes, and
population-scale pangenomic data from the 100,000 Genome Project contains the
genomes of $\approx85$K individuals. Existing data structures and software
tools fail to scale to these data sizes, limiting the value of these essential
public resources, and hampering critical data re-analysis in light of new and
improved computational methods.

The project’s novelties are: a vertical-stack approach spanning \defn{theory
and algorithms}: highly concurrent, dynamic, and distributed data structures,
\defn{systems}: scale up using GPU acceleration, \defn{high-performance
computing}: scale out using distributed data structures; new parallel and
distributed data structures and algorithms to exploit the massive compute on
GPUs applicable to other application domains; and an API for developers to
quickly and seamlessly integrate high-performance and scalable data structures
in applications.
%
Our team constitutes a highly interdisciplinary array of researchers across :
applications (computational biology), theory and algorithms, systems, and
high-performance computing. The team is taking a holistic
theory/systems/HPC/applications co-design approach to explore four tightly
interconnected research modules. These research modules are structured from
bottom-up across the computing stack.

\noindent \textbf{\large Broader Impacts: }\\
The primary broader impact of this work is to enhance the capability of
bioinformatics applications to perform computations at the largest scales. Both
accelerated computation (allowing quicker feedback and more experiments) and
larger computation potentially accelerate the process of scientific discovery.
Second, we expect that a high-quality library of distributed, scalable GPU data
structures will find widespread utility in other application domains.
%
Beyond technical broader impacts, we propose outreach impacts through tutorial
on using our data structure library in computational biology conferences and
proposing week-long research seminars like Dagstuhl that we hope will bridge
the gaps between computational biologists, CS theorists, and CS systems
researchers. 
