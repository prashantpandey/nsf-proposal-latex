%!TEX root =  proposal.tex

%% Don't actually need this - will submit txt.  Just want to estimate actual space
\begin{center}

\bf
\Large
Collaborative Research: CSR\@: Medium: \\Dynamic and Distributed Data Structures on the GPU

% \medskip
\small
Lead PI\@: Mart\'{\i}n Farach-Colton (NYU) \\
Co-PIs: Michael A. Bender (SBU), John Owens (UC Davis), Prashant Pandey (Northeastern), Rob Patro (UMD)
\end{center}

\vspace{-0.3cm}

\noindent \textbf{\large Overview:}\\
The volume and variety of available data in need of computational processing are growing at an increasing pace, driven by our ever increasing capacity to rapidly generate information across various fields such as  computational biology, astrophysics, quantum chemistry, social networks, IoT devices, etc. Applications in these areas often need to process data at petabyte scales. 
GPUs are cost-effective and offer massive parallelism, allowing significant speedups compared to CPUs. 
%GPU can help speed up large-scale data analyses tasks. 
However, a wide range of data-analysis tasks are bottlenecked on the GPU performance of common data structures.  Without a principled redesign of data structures, additional device RAM and GPU cores will yield diminishing returns.

This project aims to develop high-performance and scalable data analysis-pipelines for modern data-intensive applications. Specifically, we aim to develop new massively-parallel (GPU) and distributed data structures and algorithms that sit at the heart of many data processing tasks. These new data structures and algorithms will have wide applicability across various domains.
%
We will develop new \defn{algorithmic theory} to design scalable data structures and new \defn{systems} that implement our solutions in a scale-up manner on GPUs. We will develop a new framework to distribute our solutions across nodes, so they scale out to clusters of GPUs; and finally we will validate our solutions on \defn{computational biology} workloads.
%
Specifically, the data structures we propose to study are: \defn{filters}, \defn{sketches}, \defn{hash tables}, \defn{string indexes}, \defn{succinct data structure primitives}, and related data structures. We will validate our solutions on \defn{taxonomic classification for metagenomic data}, \defn{large-scale raw sequence search}, \defn{pangenome indexing and alignment} and others.

\noindent \textbf{\large Keywords:} GPUs, performance engineering, data structures, algorithms,   computational biology.

\noindent \textbf{\large Intellectual Merit:}\\
If successful we will be able to quickly perform complex data analyses on terabyte- and petabyte-scale datasets. For example, raw sequencing data from sequence read archive (SRA) is already at petabyte scale, metagenomic data from Western Arctic and Rhizo are in terabytes, and population-scale pangenomic data from the 100,000 Genome Project contains the genomes of $\approx85$K individuals. Existing data structures and software tools fail to scale to these data sizes, limiting the value of these essential public resources, and hampering critical data re-analysis in light of new and improved computational methods.

The project’s novelties are: a vertical-stack approach spanning \defn{design and analysis of data structures}: highly concurrent, dynamic, and distributed data structures,
\defn{systems}: scale up using GPU acceleration, \defn{performance engineering}: scale out using distributed data structures; new parallel and
distributed data structures and algorithms to exploit the massive compute on
GPUs applicable to other application domains; and an API for developers to
quickly and seamlessly integrate high-performance and scalable data structures
in applications.
%
Our team constitutes a highly interdisciplinary array of researchers across design and analysis of data structures, systems, performance engineering, and computational biology. The team is taking a holistic data structures/systems/performance engineering/applications co-design approach to explore four tightly interconnected research modules. These research modules are structured from bottom-up across the computing stack. The PIs further plan to collaborate with researchers from Nvidia to increase the adoption of the data structure in real-world as indicated in the letters of collaboration.


\noindent \textbf{\large Broader Impacts: }\\
The primary broader impact of this work is to enhance the capability of
bioinformatics applications to perform computations at the largest scales. Both
accelerated computation (allowing quicker feedback and more experiments) and
larger computation potentially accelerate the process of scientific discovery.
Second, we expect that a high-quality library of distributed, scalable GPU data
structures will find widespread utility in other application domains.
%
Beyond technical broader impacts, we propose outreach impacts through tutorial
on using our data structure library in computational biology conferences and
proposing week-long research seminars like Dagstuhl with the goal of bridging
the gaps between computational biologists, CS theorists, and CS systems
researchers. 
