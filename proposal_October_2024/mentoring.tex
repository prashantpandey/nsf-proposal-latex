% -*- Mode: Latex -*-
%\documentclass[11pt,letterpaper]{article} 
% It turns out we can use 10-point if use palatino

%!TEX root =  proposal.tex

\usepackage[T1]{fontenc}
%\usepackage[latin1]{inputenc}
\usepackage[amssymb]{SIunits}
\usepackage{latexsym}
\usepackage{multirow}
\usepackage{wrapfig}
\usepackage{amsmath, amsthm, amssymb}
\usepackage{amsfonts}
\usepackage[format=plain,indention=0cm, font=small, labelfont=bf]{caption}
\usepackage{longtable}
\usepackage{fancyhdr}
\usepackage[pdftex]{graphicx}
\usepackage[pdftex,
	    colorlinks,
	    pdfstartview=FitH,
	    linkcolor=black,
	    citecolor=black,
	    urlcolor=black,
	    filecolor=black
	    ]{hyperref}
\usepackage{lscape}
\usepackage{floatrow}
%\usepackage{enumerate}
%\usepackage{enumitem}
\usepackage{tabularx}
\usepackage{ragged2e}

\usepackage[normalem]{ulem} % allows text to be striked out. use \sout{xxx}

% \definecolor{coolblack}{rgb}{0.0, 0.18, 0.39}
% \definecolor{darkcerulean}{rgb}{0.03, 0.27, 0.49}
% 
% \newcommand{\propose}[1]{{\color{darkcerulean}#1}}

\usepackage[table]{xcolor}
\usepackage{comment}
\usepackage{ifthen}

\newcolumntype{Y}{ >{\RaggedRight\arraybackslash}X}
\newcommand\T{\rule{0pt}{2.6ex}}
\newcommand\B{\rule[-1.2ex]{0pt}{0pt}}

\renewcommand{\refname}{References Cited}

\newcommand{\degrees}{$\!\!$\char23$\!$}

\def\rrr#1\\{\par
\medskip\hbox{\vbox{\parindent=2em\hsize=6.12in
\hangindent=4em\hangafter=1#1}}}

\renewcommand\baselinestretch{1}
%%% \pagestyle{empty}
%%% INCLUDE THE ABOVE TO PUT PAGE NUMBERS BACK IN

%\setlength{\parindent}{0pt}
\setlength{\parskip}{1ex}

%!TEX root =  proposal.tex

\usepackage[moderate,bibnotes,charwidths,mathdisplays,indent]{savetrees}

%\usepackage[textsize=tiny]{todonotes} % package for cool todonotes in the margin
%\usepackage[normalem]{ulem} % package that allows text to be striked out
\usepackage{titlesec}

\titlespacing*{\section}
{0pt}{2.5ex plus 1ex minus .2ex}{1.3ex plus .2ex}
\titlespacing*{\subsection}
{0pt}{2.5ex plus 1ex minus .2ex}{1.3ex plus .2ex}


\usepackage{amsmath}
\usepackage{amsthm}
\usepackage{amstext}
\usepackage{amsfonts}
\usepackage{mathrsfs}
\usepackage{xcolor}
\usepackage{tikz}
\usepackage{cleveref}
\usepackage{subcaption}
\usepackage{cite}
\usepackage{pgfplots}
\pgfplotsset{compat=1.3}
\usepackage{url}
\urlstyle{tt}
\usepackage{xspace}
%\usepackage{algorithm}
%\usepackage[noend]{algpseudocode}
%\usepackage{siunitx}
\usepackage{booktabs}
\usepackage{multirow}
\usepackage{comment}
\usepackage[title]{appendix}
\usetikzlibrary{external}
%\usepackage{enumitem}
\usepackage{balance}
\usepackage{pgfgantt}

\iffalse
\usepackage{xr}
% In your preamble

\makeatletter
\newcommand*{\addFileDependency}[1]{% argument=file name and extension
  \typeout{(#1)}
  \@addtofilelist{#1}
  \IfFileExists{#1}{}{\typeout{No file #1.}}
}
\makeatother

\newcommand*{\myexternaldocument}[1]{%
    \externaldocument{#1}%
    \addFileDependency{#1.tex}%
    \addFileDependency{#1.aux}%
}
\myexternaldocument{supplementary}
\fi



\usepackage{graphicx}

\usepackage{svg}

%\usepackage{amssymb}

%!TEX root =  proposal.tex


%% MAB: for proposing text
\definecolor{coolblack}{rgb}{0.0, 0.18, 0.39}
\definecolor{darkcerulean}{rgb}{0.03, 0.27, 0.49}
\newcommand{\propose}[1]{{\color{darkcerulean}#1}}

% MAB: this is the macro we can use to enumerate research problems
\newtheorem{eval}{Evalulation}
\newtheorem{prob}[eval]{Research Problem}
\newtheorem{rproblem}[eval]{Research Problem}
\newtheorem{proofofconcept}[eval]{Proof of Concept}

\newcommand{\evalref}[1]{Evaluation~\ref{#1}}
\newcommand{\evalreftwo}[2]{Evaluations \ref{#1} and~\ref{#2}}
\newcommand{\evalrefthree}[3]{Evaluations \ref{#1}, \ref{#2},             and~\ref{#3}}
\newcommand{\evalreffour}[4]{Evaluations \ref{#1}, \ref{#2}, \ref{#3},                   and~\ref{#4}}


\newcommand{\errbits}{\log (1/\epsilon)}
\newcommand{\opt}{n\errbits}
\renewcommand{\epsilon}{\varepsilon}

% \newcommand{\defn}[1]{\textit{#1}}
\newcommand{\defn}[1]{{\textit{\textbf{\boldmath #1}}}\xspace}
\newtheorem{theorem}{Theorem}
\newtheorem{lemma}[theorem]{Lemma}
\newtheorem{corollary}[theorem]{Corollary}
\newtheorem{claim}[theorem]{Claim}
\newtheorem{proposition}[theorem]{Proposition}
\newtheorem{assumption}[theorem]{Assumption}
\newtheorem{definition}[theorem]{Definition}
\newtheorem{invariant}[theorem]{Invariant}
\newtheorem{observation}[theorem]{Observation}

\ifdefined\textln\relax\else\newcommand{\textln}[1]{#1}\fi

%\makeatletter
%\algrenewcommand\ALG@beginalgorithmic{\footnotesize}
%\makeatother


\definecolor{magenta4}{rgb}{0.5625,0,0.5625}
\definecolor{green4}{rgb}{0,0.5625,0}
\definecolor{orange4}{rgb}{0.98,0.31,0.09}
\definecolor{powderblue}{rgb}{0.69,0.88,0.9}

\iftrue
\newcommand{\mab}[1]{{\scriptsize \textcolor{blue}{Michael: {#1}}}}
\newcommand{\mfc}[1]{{\scriptsize \textcolor{orange4}{Martin: {#1}}}}
\newcommand{\john}[1]{{\scriptsize \textcolor{pink}{John: {#1}}}}
\newcommand{\prashant}[1]{{\scriptsize \textcolor{cyan}{Prashant: {#1}}}}
\fi


% see http://latexcolor.com to define more colors

\newcommand{\fixme}[1]{\todo[size=\tiny]{#1}}

\newcommand{\TODO}{{\color{red}\bf TODO}}


\newcommand{\qf}{quotient filter\xspace}
\newcommand{\cqf}{counting quotient filter\xspace}
\newcommand{\bloom}{Bloom filter\xspace}
\newcommand{\btree}{B-tree\xspace}
\newcommand{\btrees}{B-trees\xspace}
\newcommand{\bplustree}{B$^+$-tree\xspace}
\newcommand{\bplustrees}{B$^+$-trees\xspace}
\newcommand{\proc}{\texttt}


\newcommand{\kmer}{$k$-mer\xspace}
\newcommand{\Kmer}{$K$-mer\xspace}
\newcommand{\kmers}{$k$-mers\xspace}
\newcommand{\Kmers}{$K$-mers\xspace}


\newcommand{\poly}[1]{\textrm{poly}{(#1)}}
\newcommand{\polylog}[1]{\textrm{polylog}{(#1)}}

\newcommand{\para}[1]{\smallskip\noindent\textbf{#1.}}

\newcommand{\punt}[1]{}

\pagenumbering{gobble}



\begin{document} 

\thispagestyle{empty}

\begin{center}
   \textbf{Mentoring Plan}\\
 Collaborative Research: CSR: Medium: Dynamic and Distributed Data Structures on the GPU\\
 PIs Michael A. Bender, Mart\'{\i}n Farach-Colton, John Owens, Prashant Pandey and Rob Patro
\end{center}

The PIs and their academic institutions are committed to advancing and developing the careers of young researchers. We will follow the mentoring plan outlined below to ensure that the graduate students hired under this project develops the research skills, knowledge, and experience necessary
to advance their career.

\paragraph{Orientation.} The PIs will hold in-depth conversations with their respective graduate student(s). To assist in the expectation setting process, students and PIs are encouraged to communicate expectations and goals at initial orientation meetings as well as at twice-annual review meetings. Northereastern, NYU, Stony Brook, UCD, and UMD all offer comprehensive student orientation programs as well as programs that provide student guidance.

\paragraph{Publications and Presentations.} The project will result in publications in conferences and journals in algorithms and applied algorithms. The PIs and graduate students will collaborate closely on the proposed research and in preparing papers for submission, through regular weekly meetings, and additional meetings with outside collaborators, as needed. In addition, the PIs and their students will meet at twice annual in-person research retreats. The PIs will provide detailed feedback and guidance on any scientific writing. A goal for the graduate students is to increase their ability to work with research in far-flung parts of computer science, so special attention will be paid to helping them master these skills.

\paragraph{Teaching and Mentoring Skills.} Graduate students will develop teaching and mentoring skills in the context of regular meetings within their research groups during which students describe their work to colleagues and assist each other with solutions to challenging research problems, often resulting in cross-fertilization of ideas.

\paragraph{Networking and Collaboration.} One of the major goals for this project is to develop new vector algorithms that can impact not only the theory of vector search, but the implementation and practice as well. Therefore collaboration with applied machine learning researchers and practitioners will be important for advancing the work in the proposal beyond TCS. To help graduate students build connections at and across their departments, the PIs will involve them in existing research and reading groups. 

\paragraph{Proposal Preparation.} Graduate students will gain direct involvement in proposal preparation by the PIs to learn best practices, including identification of key research questions, definition of objectives, description of approach and rationale, and construction of a work plan, timeline, and budget. Students will also have access to approved sample grants, as well as grant writing tips.

\paragraph{Career Counseling.} The PIs will provide a direct resource for advice on the graduate students’ research careers. Additionally, students at Northereastern, NYU, Stony Brook, UCD, and UMD have access to career counseling resources from their respective departments. Each offers programs and workshops on topics such as mentor/trainee responsibilities, collaboration in science, publication practices and responsible authorship, ownership, acquisition, storage and sharing of research data, and survival skills for a career in research.

\end{document}