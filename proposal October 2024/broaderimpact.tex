\section{Broader Impacts}
\label{broadimpacts}

The primary broader impact of this work is to enhance the capability of bioinformatics applications to perform computations at the largest scales. Both accelerated computation (allowing quicker feedback and more experiments) and larger computation potentially accelerate the process of scientific discovery. 
% As MIT professor Bonnie Berger notes (specifically with respect to computing drug-target interactions, but more generally as well)~\cite{Krasilnikova:2022:DDD}, ``the data's growing enormously. It's been very difficult to handle, and it will keep growing, but it has huge opportunities.''
%
Specifically, the ability to search through all of SRA will allow researchers to quickly answer complex biological questions such as: determine if a new putative disease-related transcript has appeared in other biological samples and determine if a new fusion event found in a cancer subtype is common among other samples. Answering these important biological questions at the SRA scale is not feasible today. This can enable researchers to detect a novel disease variant in time to stop its spread.
% 
Furthermore, the analysis of genomic variants combined with phenotypic information of samples in a pangenomic index will enable researchers to improve critical biological applications such as personalized medicine, population-level disease analysis, and cancer remission rate prediction.

% \john{Prashant: if there's a good societal benefit here, say, something about disease, could you add?}

Secondly, we expect that a high-quality library of distributed, scalable GPU data structures will find widespread utility in other application domains. Although applying our library to such domains is outside of the scope of this project, we see numerous application where this work can have a direct impact. For instance, PI Owens found that multi-GPU graph computation on GPUs without scalable data structures is largely limited by GPU-to-GPU bandwidth~\cite{Pan:2017:MGA} and has proposed complex mechanisms like asynchronous execution and on-GPU communication~\cite{Chen:2022:SIP} to alleviate this bottleneck. The distributed hash tables in Section~\ref{sec:hash-tables} can directly address this challenge.

Beyond technical broader impacts, we propose two outreach impacts that we hope will bridge the gaps between computational biologists, CS theorists, and CS systems researchers:

\begin{enumerate}%[leftmargin=*,noitemsep]
  \item We will develop a tutorial on our software and techniques that is suitable for presentation at computational biology conferences (e.g., ACM RECOMB, ISMB, or PSB) and database conferences. Our primary goal here is enabling attendees to use both our computational-biology applications and the underlying data structures library, and secondarily to understand the design decisions and research advances that underpin them.
  \item We will propose, and lead, a week-long Dagstuhl seminar to bring together researchers from these three communities (computational biology, databases, HPC), with the same specific focus as this proposal: large-scale applications in computational biology.
  \item We will propose workshops at database conferences such as VLDB and SIGMOD on large-scale biological (genomic, transcriptomic, metagenomic) data analyses. Our primary goal is make the database community aware of the scalability challenges in computational biology. 
\end{enumerate}

Finally, the PIs commit to furthering their strong records of working with students that are traditionally underrepresented in computing fields. Owens works with two UC programs, UC LEADS~\cite{UCLeads:2003:WEB} and the UC Davis McNair Scholars Program~\cite{McNair:2003:WEB}.  The PIs currently collectively advise six female Ph.D.\ and three female M.S.\ students and have a long history of supervising students and postdocs from underrepresented group.
PI Farach-Colton is Hispanic and has recruited many Hispanic students into PhD programs. PI Pandey currently advises a female student from an underrepresented group from BYU Idaho and three female M.S.\ students and has a long history of supervising students from underrepresented groups.  
% He also plans to involve with undergrads by presenting in the student clubs such as the cybersecurity club, undergrad research forum, and engineering day (a day the CoE brings high school students to campus and they tour labs).
